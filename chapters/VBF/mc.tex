
The simulated samples used in this analysis are provided by the MC15c
production campaign of the ATLAS MC Production
Group \cite{TWiki_AtlasProductionGroup}. 
Simulated samples are used for the SM Higgs boson
ggF and VBF production, and strong and electroweak \zjets{} production.

For all signal samples, the detector response is simulated with the \geantFour{}
package~\cite{Agostinelli:2002hh}.
For the \zjets{} background samples, the detector response is simulated 
with the fast-simulation
package \atlfastTwo{}~\cite{Richter-Was:683751}.
Pile-up effects are taken into account by using minimum bias events
generated with \pythia{}~\cite{Sjostrand:2014zea} where the mean number of interactions
per bunch crossing is adjusted to the data-taking period.

VBF and ggF Higgs boson samples are generated with the next-to-leading order (NLO)
generator \powheg{}~\cite{Nason:2004rx,Frixione:2007vw,Alioli:2010xd} using the
CT10 PDF set~\cite{Lai:2010vv} and interfaced with \pythia{} for parton showering
and fragmentation with the AZNLO tune.
%Alternative samples, obtained by processing the same generated events with 
%\herwig{}~\cite{Bahr:2008pv,Bellm:2015jjp}
%for parton showering and fragmentation, are used for assessing modeling uncertainties.
We use two \zjets{} samples generated with \madgraph{}~\cite{Alwall:2014hca} 
using the NNPDF PDF set~\cite{nnpdf} and interfaced with \pythia{} for parton showering
and fragmentation with the A14 tune. One sample is used to model QCD production.
In this sample electro-weak (EWK) production of $Z$
(e.g., $q+q \rightarrow W^{+} + W^{-} + qq  \rightarrow Z + qq$)
is explicitly not generated. To improve the efficiency of the sample
for our final state, two light partons (light quarks or gluons) are
required in the final state ($pp \rightarrow Z+ q/g + q/g$).
The second sample contains exclusively electroweak $Z$ production.
The cross sections of MC samples we use in this analysis are presented Table. \ref{tab:xsec}.

\begin{table}[]
\centering
\caption{MC sample cross section times branching ratio}
\label{tab:xsec}
\begin{tabular}{|l|l|l|l|l|}
\hline
                   & VBF $h\rightarrow b \bar b$  & ggF $h\rightarrow b \bar b$  & QCD \zjets{} & EWK \zjets{} \\ \hline
$\sigma \times$ Br (Pb) & 2.22 & 25.91 & 666.52         & 4.86         \\ \hline
\end{tabular}
\end{table}
