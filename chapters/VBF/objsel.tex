\label{sec:vbf-objsel}

The VBF \Hbb events have four jets in the final states.
The identification of the $b$-jets requies reconstruction of
{\it central} jets within the tracker coverage ($|\eta|<2.5$), while
the identification of forward VBF jets requires reconstruction
of {\it forward} jets with $2.5<|\eta|<4.4$. 


\subsubsection{Track and Primary Vertex Selection}

Several jet properties are computed using tracks associated to the jets. The tracks must pass the standard Loose track quality selection, namely:

\begin{itemize}
\item \pT $>$ 500 \mev
\item $|\eta| < $ 2.5
\item At least 7 silicon (pixel + SCT) hits
\item The sum the number of pixel hits shared with another track and half the number of SCT hits shared with another track must be less than 2
\item At most 2 missing silicon hits
\item At most 1 missing pixel hit
\item Track must be associated to the primary vertex, or $|z_0^{\rm PV}sin(\theta)| < 3$ mm  where $z_0^{\rm PV}$ represents the $z$ position of the track with respect to the primary vertex.
\end{itemize}

The track to jet matching is performed using ghost-association~\cite{GhostMatching,jetareas,atlasjetsub}, which provides a robust matching procedure that makes use of the area of the jet~\cite{GhostMatching}. In this procedure, the track four-vectors are used with track jet \pT set to an infinitesimal amount (the ``ghosts"), and re-clustering is performed using the R=0.4 anti-$k_t$
algorithm.  Tracks which are clustered into the jet are then considered to be associated to the jet.  
The primary vertex is defined as the vertex in the event with the highest $\Sigma$ \pT$^2$ of the tracks associated to the vertex and at least two tracks.


\subsubsection{Jets}
\label{sec:vbf-jets}

%
%Jets are clustered with the anti-$k_t$
%algorithm~\cite{Cacciari200657,Cacciari:2008gp} (R=0.4) starting from
%topological clusters built from energy deposited in the
%calorimeter (AntiKt4EMJets). Jets are calibrated using a 
%simulation-based, energy and $\eta$ dependent
%calibration scheme, with in-situ corrections from data.~\cite{JESwiki}.
%
%The jet energy scale (JES) uncertainty is determined using residual
%uncertainties after the in-situ correction, with additional
%uncertainties including high-p$_T$ extrapolation and pile-up
%contributions.
%
%Jet quality criteria are applied to identify so-called~\textit{bad
%jets}, those not produced by in-time real energy deposits in the
%calorimeters, but instead caused by various sources including hardware
%problems in the calorimeter, LHC beam-gas interactions, and cosmic-ray
%induced showers. The jet quality criteria used for rejecting such jets
%follow the~\textit{Loose} selection detailed
%in~\cite{ATLAS-CONF-2015-029}. 

Jet reconstructions and calibrations follow the description in Chapter \ref{chap:reconstruction}.
This analysis uses exclusively anti-$k_t$ R=0.4 jets.
Jets with $\pt < 60$ \GeV~and $|\eta| <$ 2.4 need to satisfy the 
\textit{Default} Jet Vertex Tagger selection to ensure their origins are hard scattering vertices.
All jets are required to have $\pt>20$ \GeV and $|\eta|<4.4$.


\subsubsection{\bjets}

\label{sec:vbf-btagging}
The {\it MV2c10}-tagger as explained in Chapter \ref{chap:reconstruction} is used to identify \bjets.

%weights of three \btagging algorithms (JetFitter, IP3D and SV1) as
%well as $\pt$ and $\eta$ of the jet as an input to a neural network
% to determine a final tagging discrimination weight for each
%jet.
%Further details about the \btagging algorithms used can be
%found in \cite{Aad:2015ydr,ATL-PHYS-PUB-2016-012}.

Three different {\it MV2c10}-tagger operating points are considered, respectively with 60\% (tight), 70\% (medium) and 85\% (loose) tagging efficiency for jets
with $\pt > 20~\GeV{}$ and $\eta < 2.5$ as measured in semi-leptonic $t\bar{t}$ events. The corresponding light jet rejections vary from 1204 to 84 and the charm jet rejections vary from 6 to 150.

%The corresponding light jet, charm jet and tau rejection factors are detailed in Ref.~\cite{MV2c10wiki}.
%overall tagging efficiency of 70$\%$ and a
%corresponding light jet rejection factor of 381, charm jet
%rejection factor of 13 and a tau rejection factor of \btag
%55 \cite{MV2c10wiki} for jets with $\pt > 20~\GeV{}$ and $\eta < 2.5$. This working point is measured in a sample of \ttbar events.

In order to correct for differences in the \btagging performance between the data and MC simulation, per-jet scale factors are applied to the simulated jets and their product is taken as an event weight.
%\footnote{Currently, https://twiki.cern.ch/twiki/bin/view/AtlasProtected/BTagCalib2015\#Recommendation\_November\_2016 .  This will be updated before unblinding.}  

\subsubsection{Jet track distributions for quark/gluon separation}
\label{sec:vbf-qgtagging}

The VBF jets in signal are expected to originate from the fragmentation of a light quark, 
while in background they come from the fragmentation of ISR or FSR gluons.

The track multiplicity in the jet, \ntrk, is exploited to discriminate between quark- or gluon- initiated jets in the BDT, as gluon has larger color factor and hence larger average number of \ntrk. A dedicated study~\cite{qgtagging}  provides MC calibrations for the $n_{\rm track}$ distribution, as well as shape-based uncertainties.

%Gluons have a larger QCD color factor, and consequently their shower evolution produces more splittings. This property manifests itself in a larger per-jet particle multiplicity and a wider angular spread of the jet consituents.
%The $n_{track}$-based quark/gluon tagger~\cite{qgtagging} is used for jets with $\pT>50 \GeV$ and $|\eta|<2.1$.
