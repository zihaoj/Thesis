Multiple strategies were considered for the treatment of \zjets{} contribution. Ultimately, strategy 4 was chosen. Note that these approaches were studied in prior iteration of the analysis where the BDT regions were as described in Appendix~\ref{sec:app-alternative2cen}.

\begin{enumerate}

\item Use variations in generator, parton distribution functions and the renormalization and factorization scales in the \zjets{} Monte Carlo generation to derive theoretical uncertainties which account for the potential BDT shape uncertainty. We lack the systematics variation samples ($\alpha_s$, renomalization scale) to compute the theoretical uncertainties. Nor do we have \zjets{} samples from other generators. This approach is not feasible for the current round of analysis. 

\item Use data to MC \zmujets{} as scale factor for \zjets{} BDT shape: Since the BDT input variables are largely independent of the $Z$ decay process, we can select a pure \zmujets{} sample in data and MC to determine the accuracy of modeling of the BDT shape for $Z$ events as well as the total cross section of $Z$ production in data after the pre-selection. This method would be valid if we also observe a ratio close to unity of $\frac{\zjets{}(MC)}{\zmujets{}(MC)}$.  

This approach is detailed in Appendix~\ref{sec:app-zmm}.  Cuts are applied to the dimuon $Z$-candidate to emulate the di-\bjet selection of the Higgs candidate. The cross-section ratios of data to MC after pre-selection are calculated to be 0.81$\pm$0.06 and 0.75$\pm$0.06 for \twocentral~and \fourcentral~channels respectively. The errors include the statistical uncertainties on the data and MC. 

The relative ratios of the number of events in data and MC, and between \zjets{} and \zmujets{} MC are shown in Table~\ref{tab:z_ratios} for all of the regions considered in the analysis. Notably, the \zmujets{} to \zjets{} MC ratio is not close to one in some regions, which may be due to the fact that different generators are used for the electro-weak production events, or due to residual differences in the \bjet{} and muon events.  These difference could arise from \pT-dependent \btagging effects and different final state QCD radiation effects. Hence applying the scale factors we derived from \zmujets{} study may not be directly applicable to \zjets{}. 

\item Use two-point uncertainty for \zjets{} process: The difference between using the data-derived \zmujets{}ratios and the nominal \zjets{} MC could be taken as the uncertainty.  %From the injection study (Appendix~\ref{sec:app-zmm}) one could also quote an uncertainty of $mu_H$ due to the variation of $Z$ contribution independent of other systematics. However the method can not profile the $Z$ systematics in the likelihood fit. 

\item \label{item:z-treat-4} Float the contribution of $Z$ in all BDT regions: Assume no prior knowledge of $Z$ contribution in all BDT regions and allow the contribution to float in the profile likelihood fit. This is the most conservative strategy and yields the largest decrease in overall sensitivity. 

\item \label{item:z-treat-5} Float the contribution of $Z$ in SR I and SR II of \twocentral channel and the total cross section of $Z$; Constrain the $Z$ shape variation to 50\% of the \zjets{} MC for all other BDT regions.   %Besides the SR I and SR II of the \twocentral channel, all other BDT regions we have reasonable constraints as shown in Table~\ref{tab:z_ratios}. 
As shown in Table~\ref{tab:z_ratios}, the regions other than SR I and SR II of the \twocentral channel, have good closure between \zjets{} and \zmujets{} MC as well as small difference across \zmujets{} MCs. Therefore we can make the assumption that the \zmujets{} data to MC ratios are applicable to \zjets{}. We also know the ratios $\frac{\zmujets{}(Data)}{\zmujets{}(MC)}$ are close to one, hence we can quote the difference between 1 and $\frac{\zmujets{}(Data)}{\zmujets{}(MC)}$ as systematics. To be conservative, we quote 50\% shape variations for all these regions. This method yields smaller systematics than the strategy of floating all $Z$ contributions and better reflects our knowledge of the $Z$ contribution.

\end{enumerate}



\begin{table}[]
\centering
\caption{Relative Ratios of $\frac{\zmujets{}(Data)}{\zmujets{}(MC)}$, 
$\frac{\zjets{}(MC)}{\zmujets{}(MC)}$ 
and $\frac{\zmujets{}(\textnormal{Powheg})}{\zmujets{}(\textnormal{Madgraph})}$}
\label{tab:z_ratios}
\begin{tabular}{|l|r|r|r|}
\hline
Region       &  $\frac{\zmujets{}(Data)}{\zmujets{}(MC)}$ & $\frac{\zjets{}(MC)}{\zmujets{}(MC)}$ & $\frac{\zmujets{}(\textnormal{Powheg})}{\zmujets{}(\textnormal{Madgraph})}$  \\ \hline
SR I   (2cen)  & 0.73 (0.14) & 0.35 (0.10) & 1.06 (0.28)\\ \hline
SR II  (2cen)  & 0.50 (0.08) & 0.54 (0.10) & 1.40 (0.35)\\ \hline
SR III (2cen)  & 0.91 (0.09) & 1.03 (0.12) & 0.95 (0.14)\\ \hline
SR IV  (2cen)  & 1.16 (0.07) & 1.14 (0.08) & 0.97 (0.09)\\ \hline
SR I   (4cen)  & 0.82 (0.06) & 0.74 (0.08) & 1.00 (0.11)\\ \hline
SR II  (4cen)  & 0.83 (0.05) & 0.95 (0.09) & 0.89 (0.08)\\ \hline
SR III (4cen)  & 0.96 (0.03) & 1.05 (0.04) & 1.07 (0.06)\\ \hline
SR IV  (4cen)  & 1.06 (0.02) & 1.01 (0.03) & 0.99 (0.04)\\ \hline

\end{tabular}
\end{table}


