Differentiating quark-initiated (quark) jets from gluon-initiated (gluon) jets has broad applicability to searches and measurements at the LHC.  Identifying the nature of a jet through its internal structure has a long history, originating with the discovery of the gluon at PETRA~\cite{Bartel:1979ut,Berger:1979cj,Barber:1979yr,Brandelik:1979bd}.  Recent interest has resulted from an enhanced theoretical~\cite{Larkoski:2014pca,Gras:2017jty,Frye:2017yrw}, phenomenological~\cite{Gallicchio:2011xq,Gallicchio:2012ez}, and experimental~\cite{Aad:2014gea,ATLAS-CONF-2016-034,ATL-PHYS-PUB-2017-009,CMS-PAS-JME-13-002,CMS-DP-2016-070} understanding of quark-versus-gluon jet tagging as well as the development of powerful machine learning techniques that can utilize the entire jet internal radiation pattern~\cite{Komiske:2016rsd,Dery:2017fap}.  The key difference between quark and gluon jets is that the former carry only one quantum chromodynamic (QCD) color while the latter have both a color and anti-color.  More precisely, the Altarelli-Parisi splitting functions~\cite{Altarelli:1977zs} contain a factor of $C_A=3$ for gluon radiation from a gluon and a factor of $C_F=4/3$ for gluon radiation from a quark.  Due to this difference, gluon jets tend to have more constituents and a broader radiation pattern than quark jets.  All experimental studies so far have focused on combining a small number of key jet observables that capture these expected differences.  The purpose of this note is to present a first full detector simulation study of quark versus gluon jet tagging using the entire radiation pattern inside a jet.  The approach is based on state-of-the-art image classification techniques and is benchmarked against classical quark versus gluon jet tagging schemes.  A complete comparison with a (physically-motivated) dimensionally reduced set of inputs is beyond the current scope.
