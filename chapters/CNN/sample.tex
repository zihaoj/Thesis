The generic dijet events which include abundant quark and gluon jets generated with \textsc{Pythia} 8~\cite{Pythia,Pythia8} using the A14 tune~\cite{ATL-PHYS-PUB-2014-021}, the NNPDF2.3~\cite{Ball:2014uwa} PDF set, and processed with a full simulation of the ATLAS detector~\cite{Agostinelli:2002hh,Aad:2010ah} is used for this study.  Additional samples generated with \textsc{Sherpa} 2.1.1 (CT10 PDF~\cite{Gao:2013xoa}) and \textsc{Herwig++} 2.7.1 (UE-EE5 tune~\cite{Seymour:2013qka} and CTEQ6L1 PDF~\cite{Stump:2003yu}) are used to quantify the model dependence. Both \textsc{Pythia} 8 and \textsc{Herwig++} is interfaced with \textsc{EvtGen} v1.2.0~\cite{EvtGen} for heavy flavor decays.

As quarks and gluons carry color charge and jets are color neutral, 
there is some ambiguity in the labeling of jets in simulation as quark or gluon.
In this study, jets in simulation are 
labeled based on the highest-energy parton emerging from the hard-scatter collision within the jet catchment area~\cite{area}, 
just as was used and studied in previous studies~\cite{ATL-PHYS-PUB-2017-009}.
Only jets labeled as gluon or light quark (i.e. excluding bottom and top quark) are considered.
