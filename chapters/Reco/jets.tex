Energy deposits in calorimeter cells are first grouped into 3-D clusters called topological clusters (topocluster)\cite{PERF-2014-07}. We first form proto-clusters at the EM energy scale. The cells which pass a signal to noise significance $E/\sigma$  threshold are considred to belong to a cluster. The clustering algorithm starts with the seeds of significance$>4$ and then collects topologically connected (in the same layer adjacent or in two adjacent layers but have $\eta- \phi$ overlap) cells which are sig$>2$ and their direct neighbors with sig>0. Large clusters may lead to bad jet resolutions and are hence splitted if >=2 local maxima exists with $E^{EM}>500$ MeV with some spatial and layer constraints.

Many jet clustering algorithms exist. The most widely used algorithm within ATLAS is a sequential algorithm called $anti-kt$\cite{Cacciari:2008gp}. The jet shaped formed by this algorithm is not affected by soft radiations and making all the jet objects as conical. The algorithm treats clusters as input pseudo jets and calculates two metrics $d_ij$ and $d_iB$. If a pseudo jet is closer to the beam than to any other particle by these metircs then it is left alone. Otherwise it is combined with its nearest neighbor as as a single pseudo-jet adding together their momenta. The clustering algorithm continues untill no pseuodo jets can be combined.

As hadrnoic objects, jets require more dedicated correction of its four momenta than any other objects. ATLAS apply a series corrections\cite{PERF-2016-04} to jets in steps. First $|\eta|$ resolution is improved by using event primary vertex as jet origin. Pile-up contamination is corrected in the next step. Then MC based four momenta correction as well as jet initial composition corrections are applied. Jets in data also receive in situ callibration which corrects the difference between MC and data.

