\label{sec:reco-tracking}
Many algorithms and software designs are available for tracking. For details readers may refer to \cite{Cornelissen:1020106}. The following presents the reader with a brief summary of primary algorithms used by ATLAS as the tracking reconstruction procedure, the so called ``inside-out'' procedure. Track finding starts from the precision detectors pixel and SCT. We first find 3-D representation of pixel and SCT clusters/hits called ``space points''. Then track seed searches are done with three pairs ``space points'' with $z$ constraints (tighter cut on the primary vertices, which are the interaction points of $p\bar p$ collisions) or no constraints (loose cut on primary vertices.) Track seeds at this point contain directional information. The track fitting utilizes Kalman filter assuming the hit sequence to follow Hidden Markov Model. While the filter moves along the track seeds direction, it incorporates the new hits it sees and updates the track fit with five helix parameters at the perigee of the track. Only 10\% of the seeds eventually lead to real tracks. To resolve the ambiguity arising from seeded tracking, track candidates are ranked by their quality scores. Scores favor tracks with high $\chi^2$, less holes and hits from higher resolution part of the detector. Shared hits are then assigned to higher score tracks and lower score tracks are refitted. For very dense environments, in which the pixel clusters may merge, see for example Fig.\ref{fig:reco-trackingcluster}, neural network tools are utilized to separate apart clusters created by no more than two individual charged particles\cite{PERF-2012-05,Aaboud:2017all}. The track segment found by the pixel and SCT collaboratively are then extended to include matched TRT hits. The alignment, which refers to accurate determination of the tracking detector geometry, of the tracker needs to be performed online to reduce hit to track residuals as the thermal condition of the pixel detector changes during data taking. Especially during the early Run II operation, the IBL was observed to have an unexpected bowing giving rise to an up to 6$\mu m$ vertical shift and needs real time corrections.\cite{tracking-align}.

\begin{figure}[htpb!]
\begin{center}
  \includegraphics[width=0.55\linewidth]{figures/Reco/TrackingClusterB}
  \caption{ Illustration a merged pixel cluster due to collimated charged particles originating from high energy parent (Ref.\cite{Aaboud:2017all}). }
\label{fig:reco-trackingcluster}
\end{center}
\end{figure}


Tracks used by physics analyses need to pass quality and kinematics cuts. The loose track quality\cite{ATL-PHYS-PUB-2015-018} is defined as:

\begin{itemize}
\item \pT $>$ 400 \mev
\item $|\eta| < $ 2.5
\item At least 7 silicon (pixel + SCT) hits
\item At most 1 shared module, which are modules (1 hit for pixel and 2 hits for the same layer SCT) shared between two reconstructed tracks
\item At most 2 silicon holes, which are charged particle interaction with the tracker materials but left on hits on the tracker 
\item At most 1 pixel holes
\end{itemize}

On top of the criteria of loose tracks, the tight track criteria\cite{ATL-PHYS-PUB-2015-018} include also the following:

\begin{itemize}
\item No pixel holes
\item At least 1 hit on IBL or B-layer
\item Silicon hits $\geq$ 9 (if $|\eta|\leq 1.65$)
\item Silicon hits $\geq$ 11 (if $|\eta|\geq 1.65$)
\end{itemize}

The reconstruction efficiency of tracks depend on the type of decay, initial particle momentum and psuedorapidity and so on. Fig. \ref{fig:reco-trackingeff} shows the per-track tracking efficiency as a function of parent particle \pt in different decay modes. Particularly relevant for this thesis is the efficiency of tracking for $B$ mesons, which with loose track requirement stays reasonably above 85\% up to 1\TeV. When the pile-up contribution goes up, accidental combinations of hits could also pass all the cuts we apply to the track candidates and yield so called ``fake'' tracks. This effect is estimated to be $\leq 5\%$ for the loose category tracks with $\mu = 30$ as shown in Fig.\ref{fig:reco-trackingfake}.  The illustration of track parameters including $\sdip$ are presented in Fig.\ref{fig:reco-trkdef}. The $d_0$ and $z_0$ parameters and their associated significance are very important for identification of $b$-hadrons and will be discussed more in details in Chapter \ref{chap:btagging} and Chapter \ref{chap:gbb}.


\begin{figure}[htpb!]
\begin{center}
  \includegraphics[width=0.55\linewidth]{figures/Reco/TrackingEfficiency}
  \caption{Per-track tracking efficiency as a function of \pt of the parent particle which decays before the IBL for $\rho$, three- and five-prong $\tau$ and B0 (Ref.\cite{Aaboud:2017all})}
\label{fig:reco-trackingeff}
\end{center}
\end{figure}

\begin{figure}[htpb!]
\begin{center}
  \includegraphics[width=0.55\linewidth]{figures/Reco/TrackingFake}
\caption{An estimation of the tracking fake rate, assuming the real tracks scale linearly as a function of $\mu$ for different track categories (Ref.\cite{ATL-PHYS-PUB-2015-051}). }
\label{fig:reco-trackingfake}
\end{center}
\end{figure}

\begin{figure}[htpb!]
\begin{center}
  \includegraphics[width=0.4\linewidth]{figures/Reco/trackpar.png}
\caption{Tracking parameter definitions. Tracks are parameterized by their five helix parameters. The $d_0$ and $z_0$ are the transversal and longitudinal impact parameters. The $\sigma(d_0)$and $\sigma(z_0)$ are defined as the uncertainty of the impact parameters. The $\sdip=\frac{d_0}{\sigma(d_0)}$ and $\szip=\frac{z_0}{\sigma(z_0)}$ are defined as the significance of the impact parameters.}
\label{fig:reco-trkdef}
\end{center}
\end{figure}


The reconstructed tracks can help us find the event primary vertex, where the proton-proton hard scattering takes place. The set of tracks which are considered for vertices finding need to pass \cite{ATL-PHYS-PUB-2015-026}:

\begin{itemize}
\item $\pT > 400 \mev$
\item $|\eta|<2.5$
\item Number of silicon hits$\geq$9 if $|\eta|\leq 1.65$
\item Number of silicon hits$\geq$11 if $|\eta|>1.65$
\item At least 1 hit on IBL or B-layer
\item At most 1 shared module
\item No pixel holes
\item At most 1 SCT holes
\end{itemize}

The vertices finding algorithm, documented in \cite{PERF-2015-01}, starts with a seed position which is the mode of the z coordinates of all tracks. Knowing the seed, the algorithm iteratively calculates the vertex position by down-weighting the tracks which are far away and not compatible with this vertex. Each iteration re-calculates the vertex position. Tracks which are 7 $\sigma$ away are considered to not belong to this vertex and pooled for the iteration of finding the next vertex. Among all the vertices considered with at least two tracks associated to them, the one with highest track $\sum\pt^2$ is usually deemed as the primary vertex.

\begin{figure}[htpb!]
\begin{center}
  \includegraphics[width=0.55\linewidth]{figures/Reco/TrackingVertex}
\caption{Vertex reconstruction efficiency as a function number of tracks\cite{ATL-PHYS-PUB-2015-026}}
\label{fig:reco-primaryvertexeff}
\end{center}
\end{figure}


