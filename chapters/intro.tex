The search for the Standard Model (SM) Higgs boson\cite{Englert:1964et,Higgs:1964pj,Higgs:1964ia,Guralnik:1964eu}  is one of the main goals of the LHC physics program.  A new particle with mass of 125~\gev, and with  properties  compatible with those expected for the SM Higgs boson, has been discovered by the ATLAS\cite{HIGG-2012-27} and CMS\cite{CMS-HIG-12-028} Collaborations. Since its discovery, more precise measurements have strengthened the hypothesis that the new particle is indeed a Higgs boson\cite{HIGG-2013-02,HIGG-2014-06,HIGG-2014-14,CMS-HIG-12-036,CMS-HIG-12-041}.

Such a particle can be produced at the LHC through several different mechanisms. It is expected that the second largest value for the expected production cross-sections\cite{Dittmaier:2011ti}, as a function of the Higgs boson mass, is that of the vector boson fusion (VBF) mechanism $pp\to qqH$. Besides, for a Higgs boson with a mass in the range $115~\gev < m_{H} < 130~\gev$, the expected dominant decay mode is to pairs of $b$-quarks.

The thesis is organized in the following way: An introduction to the theory of the Standard Model and Higgs physics is presented in Chapter \ref{chap:theory} gives; Chapter \ref{chap:collider} gives a breif description of the Large Hadron Collider; Chapter \ref{chap:detector} describes the ATLAS detector especially its sub-systems which are relevant for this thesis; The reconstruction of physics objects used by the Higgs search and the gluon measurement are documented in Chapter \ref{chap:reconstruction}; In Chapter \ref{chap:vbf}, the main search of the Higgs boson with fully hadronic final state is presented; Improvements of \btagging and \qgtagging with modern neural networks are shown in Chapter \ref{chap:btagging} and Chapter \ref{chap:qgtagging}; The measurement of \gbb in gluon splitting dominated phase space is presented in Chapter \ref{chap:gbb}.
