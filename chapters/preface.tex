The Run I operation of the Large Hadron Collider (LHC) was so successful that both ATLAS and CMS experiments observed the existence of the Higgs boson (God Particle) through four kinds of Higgs decays. However, The Higgs boson was not observed through its highest branching ratio decay channel to a pair of bottom quarks. The LHC raised the center of mass energy to 13 TeV for its Run II operation (from 2015 to 2018) and opened a new window for exploring the properties of Higgs especially the Higgs decay to b-quarks.

This thesis focuses on searching the Higgs boson produced by Vector Boson Fusion (VBF) and decaying to bottom quarks. A search using the ATLAS detector was performed with 2016 proton-proton collision data. The multi-variate analysis measured the signal strengths of both the inclusive Higgs production and the vector-boson fusion production relative to the Standard Model prediction. This analysis led to the observation of Higgs coupling to b-quarks in the summer of 2018. Furthermore, potential improvements of the analysis techniques using complex neural networks are investigated. In order to understand better the Quantum Chromodynamics (QCD) backgrounds of the Higgs search, the characteristic variables of the gluon splitting vertex are measured.
