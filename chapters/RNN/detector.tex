
The ATLAS detector~\cite{ref:AtlasDet} is a general-purpose particle detector used to investigate 
a broad range of physics processes. It includes inner tracking devices surrounded by a 2.3 m diameter superconducting 
solenoid, electromagnetic and hadronic calorimeters and a muon spectrometer with a toroidal magnetic field. 
The inner detector consists of a high-granularity silicon pixel detector, including the insertable 
B-layer~\cite{ref:IBL} installed after Run 1 of the LHC, a silicon strip detector, and a straw-tube tracker; 
it is situated inside a 2 T axial magnetic field and provides precision tracking of charged particles with 
pseudorapidity $|\eta| <$ 2.5\footnote{ATLAS uses a right-handed coordinate system with its origin 
at the nominal interaction 
point (IP) in the centre of the detector and the $z$-axis along the beam pipe. The $x$-axis points 
from the IP to the centre of the LHC ring, and the $y$-axis points upward. Cylindrical coordinates 
$(r,\phi)$ are used in the transverse plane, $\phi$ being the azimuthal angle around the $z$-axis. 
The pseudorapidity is defined in terms of the polar angle $\theta$ as $\eta=-\ln\tan(\theta/2)$. 
The rapidity is also defined relative to the beam axis as $y = \frac{1}{2}\ln \Big( \frac{E + p_z}{E - p_z} \Big) $. }. 
The calorimeter system consists 
of finely segmented sampling calorimeters using lead/liquid-argon for the 
detection of electromagnetic (EM)
showers up to $|\eta| <$ 3.2, and copper or tungsten/liquid-argon for 
hadronic showers for 1.5 $< |\eta| <$ 4.9.  In the central 
region ($|\eta| <$ 1.7), an steel/scintillator hadronic calorimeter is used.
Outside the calorimeters, the muon system incorporates multiple layers of
trigger and tracking chambers within a magnetic field produced by a system of superconducting toroids,
enabling an independent, precise measurement of muon track momenta for $|\eta| <$ 2.7.
The ATLAS detector has a two-level 
trigger system to select events for offline analysis~\cite{ATL-DAQ-PUB-2016-001}.
