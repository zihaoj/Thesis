\subsection{RNN IP Tagger}

To understand the performance of the RNN tagging algorithm, the $b$-tagging efficiency and background rejection (1 / background efficiency) are examined inclusively and as a function of the jet $\pt$. As the RNN tagger has a four-class output, several of these outputs are combined into the following discriminant function:
\begin{equation}
D_{\mathrm{RNN}} = \ln \frac{p_{b}}{f_c p_c + (1-f_c) p_l}
\end{equation}
where $f_c$ is a parameter representing the $c$-jet fraction, which can be used change the relative importance of $c$-jet and light-jet rejection by the discriminant.  This parameter is fixed at $f_c=0.07$, which is close to the 8\% $c$-jet fraction present in the $t\bar{t}$ training sample. The tau output $p_\tau$ is ignored.

Inclusive background rejection versus signal efficiency curves are produced by scanning a minimum threshold on $D_{\mathrm{RNN}}$ and computing background rejection and signal efficiency at each threshold. These curves can be found in Figure~\ref{fig:ROC}, for a background of light jets and a background of $c$-jets separately.  Curves are also presented for IP3D, SV1, and MV2c10. One can see that the RNN outperforms other low-level taggers like IP3D and SV1. As the RNN tagger operates directly on tracks, utilizing both impact parameter and kinematic information and taking into account track-to-track correlations, it is not surprising that it outperforms the IP3D algorithm.  However, the performance is not as strong as that of MV2c10, which is expected because MV2c10 combines information from several low-level tagging algorithms, utilizing more information than is being fed to the RNN.
\begin{figure}[htbp]
  \centering
 \includegraphics[width=0.48\textwidth]{figures/RNN/BL_ROC.pdf}
 \includegraphics[width=0.48\textwidth]{figures/RNN//BC_ROC.pdf}
\caption{The light-jet (left) and $c$-jet (right) rejection versus $b$-tagging efficiency for jets with $p_T > 20$ GeV and $|\eta|<2.5$. The statistical error on the curve is less than 3\%.}
  \label{fig:ROC}
\end{figure}

In order to understand how the tagging performance depends on jet kinematics, the $b$-tagging efficiency versus jet $p_T$ is shown in Figure~\ref{fig:fixed_eff_pt}.
To isolate the effect of a changing $b$-tagging efficiency from that of the changing light-jet and $c$-jet rejection rejection, a flat-efficiency 70\% WP is examined.  In this case, all taggers will have a 70\% efficiency across $p_T$, and only the rejection is varying.  The light- and $c$-jet rejection as a function of jet $p_T$ can be seen in Figure~\ref{fig:flat_rej_pt}.  One can see that the RNN outperforms IP3D and SV1 in terms of light-jet rejection across the $p_T$ range, and outperforms IP3D in terms of charm rejection.
%% while the light-jet and $c$-jet rejection versus jet $p_T$ are shown in Figure~\ref{fig:fixed_rej_pt}.  In this figure, a threshold on the RNN output is applied using a 70\% WP.   We can see that the RNN light- and $c$-jet rejection outperforms that of IP3D for nearly all $p_T$ ranges.  The RNN $b$-tagging efficiency is close to that of MV2c10 across the $p_T$ range but does not reach the level of light- and $c$-jet rejection as MV2c10.

\begin{figure}[htbp]
  \centering
 \includegraphics[width=0.48\textwidth]{figures/RNN/BEff_FixWP70.pdf}
\caption{$b$-tagging efficiency for a 70\% WP cut versus jet $p_T$. }
  \label{fig:fixed_eff_pt}
\end{figure}

%% \begin{figure}[htbp!]
%%   \centering
%%  \includegraphics[width=0.48\textwidth]{figures/RNN/LRej_FixWP70.pdf} 
%%  \includegraphics[width=0.48\textwidth]{figures/RNN/CRej_FixWP70.pdf}
%% \caption{The Light jet rejection (left) and $c$-jet rejection (right) for a 70\% WP cut versus jet $p_T$. }
%%   \label{fig:fixed_rej_pt}
%% \end{figure}

\begin{figure}[htbp]
  \centering
 \includegraphics[width=0.48\textwidth]{figures/RNN/LRej_FlatEff.pdf}
  \includegraphics[width=0.48\textwidth]{figures/RNN/CRej_FlatEff.pdf}
\caption{Light jet rejection (left) and $c$-jet rejection (right) at flat $b$-tagging efficiency of 70\% versus jet $p_T$.}
  \label{fig:flat_rej_pt}
\end{figure}

%While the power of the RNN approach applied directly on track related variables can be seen from the efficiency and rejection curves, the source of this performance is difficult to pinpoint as this information is carried in the learned weights of the network and is not easy to interpret.  
The power of the RNN approach, which can be inferred directly from the ROC curves, is due to the deep network’s ability to build complex, non-linear, hierarchical, internal representations, and it is therefore hard to assign the performance increase to any specific input variable.
However, one can examine how the RNN discriminant $D_{\mathrm{RNN}}$ depends on the inputs to the network by examining the Pearson linear correlation coefficient, $\rho$ between the per track input variables and $D_{\mathrm{RNN}}$.  This can be seen in Figure~\ref{fig:input_output_corrs}.  The strongest correlation is the $S_{d_0}$, especially for the first  $\sim8$ tracks in the sequence.  This may be related to the typical charged particle multiplicity expected in a $b$-jet. $S_{z_0}$ is the second most correlated variable. There is a small negative correlation with the $p_T^{\textrm{frac}}$, especially for tracks later in the sequence, which may be related to the harder fragmentation of $b$-quarks compared to lighter quarks, leading to an expectation of only a small number of high $p_T^{\textrm{frac}}$ tracks. Finally there is a small negative correlation with $\Delta R$, indicating that tracks far from the jet axis are less likely inside of $b$-jets where most tracks from a high momentum $b$-hadron are collimated.

\begin{figure}[htbp]
  \centering
 \includegraphics[width=0.48\textwidth]{figures/RNN/Corr_Sd0.pdf}
 \includegraphics[width=0.48\textwidth]{figures/RNN/Corr_Sz0.pdf}
 \includegraphics[width=0.48\textwidth]{figures/RNN/Corr_pTFrac.pdf}
 \includegraphics[width=0.48\textwidth]{figures/RNN/Corr_DeltaR.pdf}

\caption{Correlations of per track input variables $S_{d_0}$ (top left), $S_{z_0}$ (top right), $p_T^{\textrm{frac}}$ (bottom left), and $\Delta R$, with the RNN score $D_{\mathrm{RNN}}$.}
  \label{fig:input_output_corrs}
\end{figure}


\subsection{Combination with ATLAS Taggers}
