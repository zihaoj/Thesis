As discussed in Chapter \ref{chap:vbf}, the VBF \Hbb measurement requires identification of \bjets for both online and offline. Improvements of \btagging techniques can surpress the light and charm backgrounds and hence increase the sensitivity of \Hbb search. More importantly, improvement for online \btagging can lower the energy threshold for \bjet triggers and hence increase the efficiency of \Hbb event selections. This chapter focuses on improving the \btagging algorithm with Recurrent Neural Network (RNN). 

The most widely used and recommended \btagging algorithm within the ATLAS collaboration during the Run II LHC operation was \textit{MV2} as described in Chapter \ref{chap:reconstruction}. As a brief reminder to the readers, the \textit{MV2} algorithm is based on three low-level $b$-tagging algorithms: \textit{IP3D} (an impact parameter(IP)-based algorithms), \textit{SV1}(an inclusive secondary vertex-based algorithm), \textit{JetFitter} (a decay chain multi-vertex reconstruction algorithm)~\cite{ref:btagPaper}. The \textit{MV2} algorithm then takes the final and intermideiate outputs from these baseline taggers as input and combines them through BDT~\cite{ATL-PHYS-PUB-2016-012}.

Many aspects of the \textit{MV2} algorithm can be improved despite its effectiveness. One could design better baseline algorithms or combination algorithm to improve \textit{MV2}. This thesis investigates particularly the improvement of the baseline algorithm \textit{IP3D}. This IP algorithm has the benefit that $b$-jet identification is possible even if no secondary vertex is explicitly reconstructed. The algorithm assigns per-track probabilities that a track is originated from a jet of a given flavor and take products of the marginal probabilities to estimate the likelihood. The per-track prbabilites are estimated by referencing binned likelihood distributions from simulation while ignoring the potential correlations between them. This simplification is driven by a practical concern in the likelihood algorithm: accounting the correlations between tracks or the different track variables would need an estimation of the joint probabilities which have unmanageably large dimensions for binned linkelihood method.

To overcome these difficulties, the Recurrent Neural Network(RNN) is adopted. RNNs are a subclass of neural networks architectures that allow extensions to sequence-based and temporal domains (see~\cite{ref:RNNthesis} and references therein). They are typically used in natural language processing~\cite{languagemodel,DBLP:journals/corr/abs-1303-5778}, machine translation~\cite{MT,MT2}, and time-series analysis~\cite{timeseries,timeseries2}. In the case of $b$-tagging, the tracks in a $b$-jet can be treated as a variable-length sequence to recast $b$-tagging into a domain where RNNs have proven to be useful. This chapter presents the work of designing a new RNN-based \btagging algorithm called \textit{RNNIP}.
