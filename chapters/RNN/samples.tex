The simulated $t\bar{t}$ and Z' production corresponding to $\sqrt{s}=$13~TeV proton-proton
collisions is used to study the tagger performance. Events are generated with the
next-to-leading order generator \powheg{}~\cite{bib:powheg} and the \textsc{CT10}~\cite{Lai:2010vv}
parton distribution functions, interfaced with \pythia{}~\cite{pythia2} for parton showing and
fragmentation. \textsc{EvtGen}~\cite{Lange:2001uf} is used to model the decays of
the $b$ and $c$-hadrons. Minimum bias interactions are generated with \textsc{Pythia8}~\cite{Pythia8}
and are overlaid on the $t\bar{t}$ events. Particles are passed through the ATLAS detector
simulation~\cite{atlas_simulation} which is based on \textsc{GEANT4}~\cite{Agostinelli:2002hh}.

Event vertex, track and jet reconstructions follow exactly the same approach as described in \ref{sec:vbf-objsel}. Events are selected by requiring a reconstructed primary vertex. The anti-$k_t$ R=0.4 jets with JVT pile-up cleaning are used for this study. Moreover, tracks are required to pass quality requirements identical to those required by \textit{IP3D}: track $p_{T} > 1$ GeV, $| d_0 | <1$ mm and $| z_0 \sin \theta | <1.5$ mm, and seven or more silicon hits, with at most two silicon holes, at most one of which is in the pixel detector,

The tracks used in the $b$-tagging algorithms are associated to jets
using the angular separation $\Delta R$ between the track and the jet axis.
The $\Delta R$ requirement varies as a function of jet $\pt$,
being wide for low $\pt$ jets and narrower for high $\pt$ jets which tend to be more
collimated~\cite{ref:btagPaper}.
A similar geometric matching scheme to match jets with truth $b$-hadrons, $c$-hadrons and $\tau$-leptons
is used to label the flavor jets as $b$, $c$, light, or $\tau$ jets in simulation as described in ~\cite{ATL-PHYS-PUB-2015-022}.

A jet is considered as $b$-tagged if the output
discriminant of the tagging algorithm
is above certain threshold. Several such thresholds,
or ``working points'' (WP), are defined, in such a way as to correspond to
a average efficiency when applied to $b$-jets from a sample of
inclusive $t\bar{t}$ events. This study focuses on the tagger performance at
70\% $b$-tagging efficiency WP, which would retain the majority of the $b$-jets.
In addition, tagger performance for flat efficiency working points will also be considered,
which are $\pt$ dependent threshold for the tagger discriminant such that the \btagging efficiency
is uniform as a function of $\pt$.
