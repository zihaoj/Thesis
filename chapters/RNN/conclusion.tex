A new low-level $b$-tagging algorithm has been presented which is built from a Recurrent Neural Network with a sequence of track-by-track variables as input.  This algorithm is seen to outperform impact parameter taggers, as is expected due to the ability to learn and discriminate on the correlations between tracks in a given jet and the ability to extend the number of input variables well beyond what is feasible with likelihood based impact parameter taggers. Given this flexibility, including more relatively low-level tracking variables as RNN inputs offers a potential avenue for further improvements. While it is difficult to pinpoint what discriminating information the RNN has learned,  this is partially illuminated by examining the correlation between the RNN output and the various track inputs to the network.

While promising, the RNN based $b$-tagger still lacks the discriminating power of higher-level algorithms like MV2. This is unsurprising given that MV2 already integrates the outputs from an IP-based algorithm with two vertex-based algorithms. Further tuning may close this gap, but more importantly the RNN discriminant need not outperform MV2 to be a useful addition to the ATLAS flavor tagging framework. Adding the RNN outputs as an input to a high-level tagging algorithm may add complementary information and result in a substantial performance boost to the high-level tagger, without any further modification of the RNN tagger. Such combinations are currently under investigation within ATLAS, which include both quantifying the performance gains from adding RNNIP  into a multi-algorithm composition of taggers like MV2 and studies of the correlations of the various low-level and high-level taggers to more fully elucidate how $b$-tagging sensitive information is used by different algorithms.