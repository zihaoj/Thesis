Particle physics research aims to answer the question what are the fundamental (inseparable) building blocks of our universe. Since the ancient Greek time, human beings have been endeavoring to answer this question and made progresses most noticeably in the $20^{\text{th}}$ century. The discovery of electron in 1897 was the prelude of a more than a hundred year long discovery relay. Due to the great advance of observational and experimental apparatus, physicists in the past century unveiled fundamental particles as light as photons, which have no mass and can be felt directly by human eyes, and those very massive such as top quarks which can only be created instantaneously and observed in powerful accelerators. Meanwhile, advanced mathematical tools helped analyze and summarize the characteristics of these fundamental particles. Through the collaborative effort of many generations of physicists, we now have established a framework called Standard Model (SM) based on quantum mechanics, a probabilistic theory. The Standard Model is a very successful theory which incorporates all the fundamental particles we know thus far and describe the interactions among them precisely. Since the birth of this theory, it has been tested extensively and successfully predicts the existence of many particles when the experimental efforts are lagging behind. Sometimes we even think the model is too successful as the attempts to search for beyond Standard Model (BSM) particles have been fruitless. The Standard Model is the causal model and guidance for this thesis. Particularly, we are interested in exploring and testing the SM descriptions of the Higgs boson and the Quantum Chromodynamics (QCD).

In this chapter, we first introduce the reader to the Standard Model in Sec.\ref{sec:theory-sm} and then focus on the discussion of the Higgs boson and its phenomenology Sec.\ref{sec:theory-higgs}. The theory discussions are adapted from the pedagogical QFT references such as \cite{peskin} and \cite{schwartz}.
