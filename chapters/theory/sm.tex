\label{sec:theory-sm}
\subsection{Overview: Particles and Interactions in Standard Model}

The Standard Model has two types of particles: Bosons and Fermions; the former have integer spins and the later have fractional spins. All particles in the Standard Model are summarized in Fig. \ref{fig:theory-sm}.

The fermions including leptons and quarks are what we normally know as matters and have spin $\frac{1}{2}$. Electrons($e^-$), muons($\mu^-$), taus($\tau^-1$) and their anti-particles ($e^+,\mu^+,\tau^+$) are leptons which carry electric charges. Associated with these charged leptons are the neutrinos ($\nu_e,\nu_{\mu},\nu_{\tau}$) and their anti-particles ($\bar{\nu_e},\bar{\nu_{\mu}},\bar{\nu_{\tau}}$).

The spin 1 bosons are the mediators of fundamental interactions or also known as force carriers. In total four types of interactions namely: strong, weak, electromagnetic(EM) and gravity. Within the Standard Model framework, since the gravity is much weaker than the rest of them, we only consider the first three kinds of forces. Among the force carriers photon($\gamma$) and gluon($g$) are massless particles and mediate the electromagnetic and strong interactions respectively. The $W^{\pm}$ and $Z$ bosons are massive particles and carry the weak force. 

All fermions can interact through week forces. Only electrically charged fermions can interact with the photon(EM force). Only the quarks carry color(the strong interaction version of charge) can interact with gluons. The Higgs boson is the only spin 0 boson within the SM and generates mass for $W^{\pm}/Z$ bosons and fermions (except neutrinos). The mechanism of mass generation is covered later in this chapter. The Higgs boson also couples to itself. 

\begin{figure}[htpb!]
\begin{center}
  \includegraphics[width=0.45\linewidth]{figures/theory/SM}
\caption{Fundamental particles within the Standard Model}
\label{fig:theory-sm}
\end{center}
\end{figure}

\subsection{Standard Model in QFT Form}

Quantum Field Theory (QFT) which is a marriage between Quantum Mechanics and Special Relativity is the modern formalism to describe the fundamental physics. In the language of QFT, fields of creation and annihilation operators will allow us to calculate the probability distributions of particles which are excited states of fields over time and space. The Standard Model is a special gauge QFT which is invariant with respect to the internal (gauge) $SU(3)\times SU(2)\times U(1)$ symmetry. The gauge symmetries, the consequence of which gives birth to the gauge bosons, are field local symmetries in addition to the global symmetries (spatial and time). 

The Lagrangian which governs the motion of the particles of SM can hence be separated into two parts: $SU(3)$ and $SU(2)\times U(1)$. The $SU(3)$ part of the Lagrangian corresponding to the strong interaction goes as in Eq.\ref{eq:l-su3}. The first term has $F^{a}_{\mu\nu}  = \partial_{\mu}G^a_{\nu}-\partial_{\nu}G^a_{\mu}+g_3f^{abc}G^b_{\mu}G^c_{\nu}$, where $G$ is the gauge field (in this case gluons), $f^{abc}$ is the structure constant of $SU(3)$ group and $g_3$ is the coupling strength for strong interaction. The second term has $\slashed{D} = \gamma^{\mu}(\partial_{\mu}- i\frac{g_3}{2}\lambda^aG^a_{\mu})$, where $\gamma$'s are the Dirac matrices, the $\lambda$'s are the $SU(3)$ matrices and $\psi$ are the quark fields. 
\begin{equation}
  \mathcal{L}_{SU(3)} = -\frac{1}{4}F^{a}_{\mu\nu}F_{a}^{\mu\nu}+ \bar{\psi}(i\slashed{D}) \psi
  \label{eq:l-su3}
\end{equation}

The $SU(2)\times U(1)$ Lagrangian has the same terms as in Eq.\ref{eq:l-su3}, replacing the gauge/fermion fields, structure constants, coupling strength and group matrices to be the corresponding ones in $SU(2)$ and $U(1)$. Note that $U(1)$ group is Abelian and hence $f^{abc}=0$ and the group matrix is a constant, while for $SU(2)$ the group matrices are the Pauli matrices ($\sigma$). In addition, the $SU(2)\times U(1)$ has terms involving an additional scalar field (complex Higgs doublet $\phi = \begin{psmallmatrix*}[r] \phi^+ \\ \phi^0 \end{psmallmatrix*} $ ) as shown in Eq.\ref{eq:l-su21}. The first term is the covariate derivative of the scalar field involving the $SU(2)\times U(1)$ gauge fields ($W/B$). The second term is the potential of the scalar field. The last term is the Yukawa couplings between the fermionic fields with the scalar field. 

\begin{equation}
  \mathcal{L}_{\phi} = (D_{\mu}\phi)^{\dagger}(D_{\mu}\phi)-(\frac{1}{2}\mu^2\phi^{\dagger}\phi+\lambda(\phi^{\dagger}\phi )^2)+Y_{ij}\bar{\psi_i}\phi\psi_j
  \label{eq:l-su21}
\end{equation}


