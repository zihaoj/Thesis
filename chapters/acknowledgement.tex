Technically I started my work on the high energy when I was an undergraduate. Since then, I have been fascinated by this sub-field of physics and the continuation of my work in this field as a graduate student was so natural a decision. In these many years, I owe my deepest gratitude to many excellent physicists and mentors. I am blessed being able to work with them and learned great amount of physics, math and statistics from them.

I shall first thank my adviser Prof. Lauren Tompkins, who I met at UChicago (just a few days before Higgs was discovered) and worked for six and half years. She was the first person who directly supervised my work and has always been mentoring me since then. She is an extremely knowledgeable physicist and excellent mentor. The support from her is tremendous on all issues including physics, funding, travel and even immigration related hassles. Besides, her great management skills as million-dollar project leader and persistence on advocating for minorities in physics and other equality/inclusion issues are constant inspirations for me not only as a physicist but also as a good person.

The Stanford ATLAS group is small but mighty. It was amazing how fast the team was built up and was able to take on major responsibilities early on in Run II for FTK. This difficult project made significant process due to the hard work of all members including Nikolina Ilic, Aaron Armbruster, Stanislava Sevova, Robert Mina and all the visiting/rotation/CSU students. I will always remember the hard days that we were down in USA15 or sitting in B4 together debugging dataflow. Thank you for being the most trustworthy teammates!

A significant portion of my thesis is a collaborative work with members of SLAC group, who are the most reliable allies. Not only did I work for the SLAC group for the summer in 2014, but also continued to collaborate with Michael Kagan, Ben Nachman, Qi Zeng and Francesco Rubbo throughout my PhD career. They contributed significantly to the analysis and performance works of this thesis and qualify as co-authors. I thank them for the great ideas they taught me, for the guidance they provided and for their care of me in general. Aside from these collaborators and mentors, I thank the generosity of the leaders of the SLAC group, Su Dong, Ariel Schwartzman and Charles Young, who allowed me to be a shameless free rider of the infrastructure and resources of the SLAC team. The superb computing cluster and professional engineering support made my work smooth, and their SLAC connections even introduced me to my next job.

Everyone I talked to in the ATLAS collaboration taught me something. I thank John Alison for introducing me to this experiment and being a great reference of everything. I thank Yasuyuki Okumura for teaching me DataFormatter firmware and going to Jia Wei in Geneva with me. I thank Tomoya Iizawa, Masahiro Morinaga, Maximilian Swiatlowski, Aviv Cukierm, Karol Krizka, Patrick Bryant, Nicole Hartman and Jannicke Pearkes for being inspiring peers. I thank Alberto Annovi, Lucian Ancu, Elizabeth Brost and Tova Holmes for their leadership in FTK. I thank the SM and Higgs group conveners for walking me through the analyses. There are just so many great and kind ATLAS people that I cannot enumerate them all.

As an immigrant, the warmest welcome from the American people came from two UChicago professors, Donald York and Melvyn Shochet. They offered me my first jobs in college and introduced to me to physics research. They are sharp physicists but more importantly the examples of kind, considerate and hard-working US people who made me feel home in this country.

Research has hard days and I am grateful to be surrounded by my family and friends, who shielded me from reality, to survive. The unconditional support from parents, the understanding and help from my girl friend Chuntian and encouraging words from all friends in China and US are my biggest source of courage in my life.

This thesis and my PhD are supported by the funding from National Science Foundation Graduate Fellowship (2016-2019) and Paul and Daily Soros Fellowship (2016-2017). I thank both organizations for relieving my financial burden and allow me to focus on research.
