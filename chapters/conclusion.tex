The Run II ATLAS experiment verfied more aspects of the SM. At the same time, there is not much hint of new physics. As the energy of LHC will remain 13-15 \tev for the rest of its lifetime, one way of looking for new physics is to measure SM particles as precisely as possible. The VBF \Hbb mentioned in Chapter \ref{chap:vbf} agreed with Standard Model prediction. However the current sensitivity of this search is low and certainly far from the point that we can perform precise measurement of this combination of Higgs production and decay mode. With the experience of this round of analysis, we have identified a few a aspects to improve the analysis.

A number of analysis specific improvements could be made in the future. Currently, the Higgs sensitivity suffered greatly from the fact that the \zbb contribution normalizations are left to float in the likelihood fit (Sec.\ref{sec:vbf-statanalysis}). Partly also due to a trigger turn-on which cuts into the $Z$ peak, around 80-90 \gev regime, the \zbb contribution is degenerate with the non-resonant background which is also parameterized with floating parameters. If we could reasonably estimate the $Z$ yield and fix its normalization while applying moderate uncertainties, the sensitivity of Higgs would be greatly improved. Potentially one could adopt the same ``embedding'' strategy as used in $h\rightarrow \tau \bar{\tau}$ analysis\cite{HIGG-2013-32}, which takes $h\rightarrow \mu \bar{\mu}$ events from data and replaces the final state $\mu$ with $\tau$. This method models the Higgs production accurately in the VBF corner phase spaces. Another aspect which could be improved is the multivariate analysis. Currently, the non-resonant background in all signal regions are modeled with independent parameters. If the  multi-variate analysis would yield same shapes for all signal regions, then all of them could be modeled with the same set of parameters. Pioneer works such as \cite{ann} has shown that Adversarial Neural Network(ANN) is a great tool for feature invariant classification problem and showed sensitivity improvement of $Z'$ search when applying ANN for large radius jet tagging

Besides, the exploration of a few programs which not only benefit the VBF \Hbb search but could also improve other SM or BSM analyses should continue in the future. The RNN based $b$-tagger which are discussed in Chapter \ref{chap:btagging} if also deployed online can help lower the threshold of $b$-jet triggers and benefit all analyses with final states containing $b$-hadrons. The CNN tagger documented in Chapter \ref{chap:qgtagging} once calibrated and also developed for the forward detector region where track varaibles such as \ntrk are not available can be used to mitigate backgrounds for all VBF Higgs analyses. Analyses such as \cite{Aaboud:2017ecz} and \cite{2h4b}, show that boosted Higgs is a pure phase space for SM and BSM searches. Such study should also be applied to VBF production mode. Jet measurements like the \gbb analysis in Chapter \ref{chap:gbb} could improve the modeling of QCD and may in future make the jet Monte Carlo precise enough for direct QCD background estimation for boosted Higgs searches. 
