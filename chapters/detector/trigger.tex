\label{sec:detector-trigger}

As already shown in \ref{fig:lhc-lhccrossection}, the Higgs processes and interesting events such as EW and the ones involving top quarks have cross sections orders of magnitude lower than the ubiquitous QCD events. Hence it is very important for us to be able to select potentially interesting physics events online and filter out unnecessary events unnecessary so that we do not swap the storage disk. A dedicated trigger system is built for ATLAS to perform online event selections quickly. Usually the trigger searches for high energy electrons, muons, taus, jets and missing transverse energy which could arise from interesting physics. The system is composed of two parts: a first level hardware based trigger (L1) and a second level software based trigger (HLT). The event rate is first reduced from $40MHz$ down to $100kHz$ by L1 then to $1kHz$ by HLT. Some of the trigger items are not run to select all events passing its thresholds on objects and select only a fraction of them (pre-scale) so that we still have access to low energy or high rate physics processes. 

The L1 triggers are separately run for the calorimeter and the muon spectrometer. The L1 calorimeter trigger aims to crudely find very large amounts of the energy deposits in bulks while the L1 muon trigger searches high momentum muons within $2.5\mu s$. L1 calorimeter triggers work with units called towers which are groups of calorimeter cells with size of $\Delta \eta \times \Delta \phi = 0.1\times0.1$. Objects which are potentially interesting based on their types should pass a certain threshold in a $2\times2$, $4\times4$ or $8\times8$ tower region of interest (RoI) to be saved. Both the RPC and TGC muon triggers have momentum measurement capability. Both of them count the number of muons passing certain $p_T$ thresholds. All L1 objects information are sent to the Central Trigger Process (CTP) to flag events passing certain items on the trigger menu and then sent to the next level. The CTP is also responsible for calculation of simple global L1 quantities such as scalar sum of or missing transverse energy and topology quantities such as $m_{jj}$, the largest invariant mass of jet pairs in an event. 

The HLT calculations are more complicated and has the time to combine full detector information and takes inputs from L1 trigger ID and also muon precision chambers. Reconstructions of vertex, tracks and etc. are performed with large CPU farms which allows us to trigger on very complicated object triggering, \bjets for instance. 
