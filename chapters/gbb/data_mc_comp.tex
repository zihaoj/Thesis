\label{sec:gbb-bkgres}
%%%%%%%%%%%%%%%%%%%%%%%%%%%%%%%%%%%%

The flavor fraction fits are performed in each bins of each observable to determine the $b\bar b $ fractions in data. 

\subsection{\drbb}

In total, there are six bins of the \drbb observable we are measuring. The flavor fraction fits are performed in $\Delta R \in\{[0.2, 0.25], [0.25, 0.3], [0.3, 0.4], [0.4,0.5], [0.5,0.6], [0.6,0.7]\}$. An example of post-fit Data/MC comparison is shown in Fig.~\ref{fig:dR-fit-example}. Refer to \ref{sec:app-subsd0fits} for all pre-fits and post-fits plots. The MC predicted and fitted fraction of each flavor component are shown in Fig.~\ref{fig:dR-fitfrac}. Four sets of fit cross check are performed: (1) using \sdzero as discriminant (2) using \subsubsdzero (3) using $p_T$ parameterized templates and (4) without track jet kinematic reweighting. The cross check results are summarized in Fig.~\ref{fig:dR-fitfrac-crosscheck}. We observe closure of the fitted signal (`BB') event fraction in all the cross checks within fitting uncertainties. The size of each of the systematic uncertainty is summarized in Table~\ref{tab:dRsys}.

\begin{figure}[htbp]
  \centering
 \includegraphics[width=0.45\textwidth]{figures/gbb/paperplots/Canv_Fit_dR_LpT_INF_SpT_INF_coarse_x}
 \includegraphics[width=0.45\textwidth]{figures/gbb/paperplots/Canv_Fit_b0_25_DeltaR_0_3_LpT_INF_SpT_INF_coarse_y}
\caption{Example of post-fit \subsdzero distributions of the leading (left) and sub-leading (right) track jets in bin of \drbb. }
  \label{fig:dR-fit-example}
\end{figure}

\begin{figure}[htbp]
  \centering
 \includegraphics[width=0.45\textwidth]{figures/gbb/paperplots/Canv_dR_FracDataMC}
\caption{MC predicted (hollow) and fitted (solid) flavor fraction in bins of \drbb. The error bar is the quadrature sum of the uncertainty derived from fit range systematics and fit statistics as defined in Sec.~\ref{sec:gbb-sub_systematics}}
  \label{fig:dR-fitfrac}
\end{figure}

\begin{figure}[htbp]
  \centering
 \includegraphics[width=0.45\textwidth]{figures/gbb/Sub_Sd0_Fits/Canv_dR_leadCrossCheck.pdf}
 \includegraphics[width=0.45\textwidth]{figures/gbb/Sub_Sd0_Fits/Canv_dR_subsubCrossCheck.pdf}\\
 \includegraphics[width=0.45\textwidth]{figures/gbb/Sub_Sd0_Fits/Canv_dR_ptbinCrossCheck.pdf}
 \includegraphics[width=0.45\textwidth]{figures/gbb/Sub_Sd0_Fits/Canv_dR_noreweightCrossCheck.pdf}\\
\caption{Comparison of flavor fractions fitted (1) using \subsdzero and \sdzero (top left), (2) using \subsdzero and \subsubsdzero (top right), (3) using inclusive and $p_T$ parameterized templates (bottom left) and (4) with and without track jet kinematic reweighting in bins of \drbb}
  \label{fig:dR-fitfrac-crosscheck}
\end{figure}


\clearpage

\subsection{\zpt}

In total, there are five bins of the \zpt observable we are measuring. The flavor fraction fits are performed for \zpt $\in\{[0, 0.1], [0.1, 0.2], [0.2, 0.3], [0.3,0.4], [0.4,0.5]\}$. An example of post-fit Data/MC comparison is shown in Fig.~\ref{fig:ZpT-fit-example}. The MC predicted and fitted fraction of each flavor component is shown in Fig.~\ref{fig:ZpT-fitfrac}. Four sets of fit cross check are performed: (1) using \sdzero as discriminant (2) using \subsubsdzero (3) using $p_T$ parameterized templates and (4) without track jet kinematic reweighting. The cross check results are summarized in Fig.~\ref{fig:ZpT-fitfrac-crosscheck}. We observe closure of fitted signal (`BB') event fraction in all the cross checks within fitting uncertainties. The size of each of the systematic uncertainties is summarized in Table~\ref{tab:ZpTsys}.


\begin{figure}[htbp]
  \centering
 \includegraphics[width=0.45\textwidth]{figures/gbb/paperplots/Canv_Fit_zpt_LpT_INF_SpT_INF_coarse_x}  
 \includegraphics[width=0.45\textwidth]{figures/gbb/paperplots/Canv_Fit_b0_25_zpt_0_3_LpT_INF_SpT_INF_coarse_y}
 \caption{Example of post-fit \subsdzero distributions of the leading (left) and sub-leading (right) track jets in bin of \zpt. }
 \label{fig:ZpT-fit-example}
\end{figure}


\begin{figure}[htbp]
  \centering
  \includegraphics[width=0.45\textwidth]{figures/gbb/paperplots/Canv_ZpT_FracDataMC}      
 \caption{MC predicted (hollow) and fitted (solid) flavor fraction in bins of \zpt. The error bar is the quadrature sum of the uncertainty derived from fit range systematics and fit statistics as defined in Sec.~\ref{sec:gbb-sub_systematics}}
  \label{fig:ZpT-fitfrac}
\end{figure}

\begin{figure}[htbp]
  \centering
 \includegraphics[width=0.45\textwidth]{figures/gbb/Sub_Sd0_Fits/Canv_ZpT_leadCrossCheck.pdf}
 \includegraphics[width=0.45\textwidth]{figures/gbb/Sub_Sd0_Fits/Canv_ZpT_subsubCrossCheck.pdf}\\
 \includegraphics[width=0.45\textwidth]{figures/gbb/Sub_Sd0_Fits/Canv_ZpT_ptbinCrossCheck.pdf}
 \includegraphics[width=0.45\textwidth]{figures/gbb/Sub_Sd0_Fits/Canv_ZpT_noreweightCrossCheck.pdf}\\
\caption{Comparison of flavor fractions fitted (1) using \subsdzero and \sdzero (top left), (2) using \subsdzero and \subsubsdzero (top right), (3) using inclusive and $p_T$ parameterized templates (bottom left) and (4) with and without track jet kinematic reweighting (bottom right) in bins of \zpt}
  \label{fig:ZpT-fitfrac-crosscheck}
\end{figure}


\clearpage
\subsection{\mpt}

In total, there are five bins of the \mpt observable we are measuring. The flavor fraction fits are performed for \mpt $\in\{[-3, -2.2], [-2.2, -1.9], [-1.9, -1.5], [-1.5, -1.1], [-1.1,0]\}$. An example of post-fit Data/MC comparison is shown in Fig.~\ref{fig:fracmasspt-fit-example}. The MC predicted and fitted fraction of each flavor component is shown in Fig.~\ref{fig:fracmasspt-fitfrac}. Four sets of fit cross check are performed: (1) using \sdzero as discriminant (2) using \subsubsdzero (3) using $p_T$ parameterized templates and (4) without track jet kinematics reweighting. The cross check results are summarized in Fig.~\ref{fig:fracmasspt-fitfrac-crosscheck}. We observe closure of fitted signal (`BB') event fraction in all the cross checks within fitting uncertainties. The size of each of the systematic uncertainties is summarized in Table~\ref{tab:fracmassptsys}.


\begin{figure}[htbp]
  \centering
 \includegraphics[width=0.45\textwidth]{figures/gbb/paperplots/Canv_Fit_M_LpT_INF_SpT_INF_coarse_x}
 \includegraphics[width=0.45\textwidth]{figures/gbb/paperplots/Canv_Fit_b0_25_M_0_3_LpT_INF_SpT_INF_coarse_y}
 \caption{Example of post-fit \subsdzero distributions of the leading (left) and sub-leading (right) track jets in bin of \mpt. }
 \label{fig:fracmasspt-fit-example}
\end{figure}


\begin{figure}[htbp]
  \centering
  \includegraphics[width=0.45\textwidth]{figures/gbb/paperplots/Canv_fracmasspt_FracDataMC}      
\caption{MC predicted (hollow) and fitted (solid) flavor fraction in bins of \mpt. The error bar is the quadrature sum of the uncertainty derived from fit range systematic uncertainties and fit statistics as defined in Sec.~\ref{sec:gbb-sub_systematics}}
  \label{fig:fracmasspt-fitfrac}
\end{figure}


\begin{figure}[htbp]
  \centering
 \includegraphics[width=0.45\textwidth]{figures/gbb/Sub_Sd0_Fits/Canv_fracmasspt_leadCrossCheck.pdf}
 \includegraphics[width=0.45\textwidth]{figures/gbb/Sub_Sd0_Fits/Canv_fracmasspt_subsubCrossCheck.pdf}\\
 \includegraphics[width=0.45\textwidth]{figures/gbb/Sub_Sd0_Fits/Canv_fracmasspt_ptbinCrossCheck.pdf}
 \includegraphics[width=0.45\textwidth]{figures/gbb/Sub_Sd0_Fits/Canv_fracmasspt_noreweightCrossCheck.pdf}\\
\caption{Comparison of flavor fractions fitted (1) using \subsdzero and \sdzero (top left), (2) using \subsdzero and \subsubsdzero (top right), (3) using inclusive and $p_T$ parameterized templates (bottom left) and (4) with and without track jet kinematic reweighting (bottom right) in bins of \mpt}
  \label{fig:fracmasspt-fitfrac-crosscheck}
\end{figure}


\clearpage
\subsection{\dphi}

In total, there are five bins of the \dphi observable we are measuring. The flavor fraction fits are performed for \dphi $\in\{[0, \pi/5], [\pi/5, 2\pi/5], [2\pi/5, 3\pi/5], [3\pi/5, 4\pi/5], [4\pi/5, \pi]\}$. An example of post-fit Data/MC comparison is shown in Figure.~\ref{fig:dphi-fit-example}. The MC predicted and fitted fraction of each flavor component is shown in Fig.~\ref{fig:dphi-fitfrac}. Four sets of fit cross check are performed: (1) using \sdzero as discriminant (2) using \subsubsdzero (3) using $p_T$ parameterized templates and (4) without track jet kinematic reweighting. The cross check results are summarized in Fig.~\ref{fig:dphi-fitfrac-crosscheck}. We observe closure of fitted signal (`BB') event fraction in all the cross checks within fitting uncertainties. The size of each of the systematic uncertainties is summarized in Table.\ref{tab:dphisys}.


\begin{figure}[htbp]
  \centering
 \includegraphics[width=0.45\textwidth]{figures/gbb/paperplots/Canv_Fit_dphi_LpT_INF_SpT_INF_coarse_x}
 \includegraphics[width=0.45\textwidth]{figures/gbb/paperplots/Canv_Fit_b0_25_dphi_0_3_LpT_INF_SpT_INF_coarse_y}
 \caption{Example of post-fit \subsdzero distributions of the leading (left) and sub-leading (right) track jets in bin of \dphi. }
  \label{fig:dphi-fit-example}
\end{figure}

\begin{figure}[htbp]
  \centering
 \includegraphics[width=0.45\textwidth]{figures/gbb/paperplots/Canv_dphi_FracDataMC}
\caption{MC Predicted (hollow) and fitted (solid) flavor fraction in bins of \dphi. The error bar is the quadrature sum of the uncertainty derived from fit range systematics and fit statistics as defined in Sec.~\ref{sec:gbb-sub_systematics}}
  \label{fig:dphi-fitfrac}
\end{figure}


\begin{figure}[htbp]
  \centering
 \includegraphics[width=0.45\textwidth]{figures/gbb/Sub_Sd0_Fits/Canv_dphi_leadCrossCheck.pdf}
 \includegraphics[width=0.45\textwidth]{figures/gbb/Sub_Sd0_Fits/Canv_dphi_subsubCrossCheck.pdf}\\
 \includegraphics[width=0.45\textwidth]{figures/gbb/Sub_Sd0_Fits/Canv_dphi_ptbinCrossCheck.pdf}
 \includegraphics[width=0.45\textwidth]{figures/gbb/Sub_Sd0_Fits/Canv_dphi_noreweightCrossCheck.pdf}\\
\caption{Comparison of flavor fractions fitted (1) using \subsdzero and \sdzero (top left), (2) using \subsdzero and \subsubsdzero (top right), (3) using inclusive and $p_T$ parameterized templates (bottom left) and (4) with and without track jet kinematic reweighting (bottom right) in bins of \dphi}
  \label{fig:dphi-fitfrac-crosscheck}
\end{figure}


\clearpage

