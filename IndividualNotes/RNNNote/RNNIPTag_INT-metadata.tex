%-------------------------------------------------------------------------------
% This file contains the title, author and abstract.
% It also contains all relevant document numbers used by the different cover pages.
%-------------------------------------------------------------------------------

% Title
\AtlasTitle{Identification of Jets Containing $b$-Hadrons with Recurrent Neural Networks at the ATLAS Experiment}

% Author - this does not work with revtex (add it after \begin{document})
%\author{The ATLAS Collaboration}

% Authors and list of contributors to the analysis
% \AtlasAuthorContributor also adds the name to the author list
% Include package latex/atlascontribute to use this
% Use authblk package if there are multiple authors, which is included by latex/atlascontribute
\usepackage{authblk}
\renewcommand\Authands{, } % avoid ``. and'' for last author
\renewcommand\Affilfont{\itshape\small} % affiliation formatting
\AtlasAuthorContributor{D.~Guest}{Ir}{}
\AtlasAuthorContributor{Z.~Jiang}{St,Sl}{}
\AtlasAuthorContributor{M.~Kagan}{Sl}{}
\AtlasAuthorContributor{M.~Paganini}{Ya}{}
\AtlasAuthorContributor{A.~Schwartzman}{Sl}{}
\AtlasAuthorContributor{P.~Tipton}{Ya}{}
\AtlasAuthorContributor{L.~Tompkins}{St,Sl}{}
\AtlasAuthorContributor{D.~Whiteson}{Ir}{}
\AtlasAuthorContributor{Q.~Zeng}{Sl}{}
\affil[Ir]{UC Irvine}
\affil[Sl]{SLAC Accelerator Laboratory}
\affil[St]{Stanford University}
\affil[Ya]{Yale}



% If a special author list should be indicated via a link use the following code:
% Include the two lines below if you do not use atlasstyle:
% \usepackage[marginal,hang]{footmisc}
% \setlength{\footnotemargin}{0.5em}
% Use the following lines in all cases:
% \usepackage{authblk}
% \author{The ATLAS Collaboration%
% \thanks{The full author list can be found at:\newline
%   \url{https://atlas.web.cern.ch/Atlas/PUBNOTES/ATL-PHYS-PUB-2014-007/authorlist.pdf}}
% }

% Date: if not given, uses current date
\date{\today}

% Draft version:
% Should be 1.0 for the first circulation, and 2.0 for the second circulation.
% If given, adds draft version on front page, a 'DRAFT' box on top of each other page, 
% and line numbers.
% Comment or remove in final version.
\AtlasVersion{0.2}

% ATLAS reference code, to help ATLAS members to locate the paper
%\AtlasRefCode{CONF-EXOT-2016-XX}

% ATLAS note number. Can be an COM, INT, PUB or CONF note
\AtlasNote{ATL-COM-PHYS-2017-102}

% CERN preprint number
% \PreprintIdNumber{CERN-PH-2014-XX}

% ATLAS date - arXiv submission; to be filled in by the Physics Office
% \AtlasDate{\today}

% arXiv identifier
% \arXivId{14XX.YYYY}

% HepData record
% \HepDataRecord{ZZZZZZZZ}

% Submission journal and final reference
% \AtlasJournal{Phys.\ Lett.\ B.}
% \AtlasJournalRef{\PLB 789 (2014) 123}
% \AtlasDOI{}
% Abstract - % directly after { is important for correct indentation
\AtlasAbstract{%
A novel $b$-jet identification algorithm is constructed with a Recurrent
Neural Network (RNN) at the ATLAS experiment at the CERN Large Hadron Collider.  The RNN based $b$-tagging
algorithm processes charged particle tracks associated to jets without reliance on secondary vertex finding, and can augment existing secondary-vertex based taggers. In contrast to traditional
impact-parameter-based  $b$-tagging algorithms which assume that tracks
associated to jets are independent from each other, the RNN based $b$-tagging algorithm can
exploit the spatial and kinematic correlations between tracks which are
initiated from the same $b$-hadrons.
This new approach also accommodates an extended set of input variables.
%% The neural network nature \todo{a BDT or SVN or anything like that would have given us the same flexibility. I would revert back to the previous version of this sentence, explaining that it's the likelihood-based nature of the current IP tagger that doesn't allow for this sort of flexibility}of the
%% tagging algorithm gives the flexibility to extend the list of input variables. 
This note presents the expected performance
of the RNN based $b$-tagging algorithm in simulated $t \bar t$ events at $\sqrt{s}=13$ TeV.
}
%and high $p_T Z^{\prime} \rightarrow q \bar q$ events where $q \in \{b,c,s,u,d\}$

%-------------------------------------------------------------------------------
% The following information is needed for the cover page. The commands are only defined
% if you use the coverpage option in atlasdoc or use the atlascover package
%-------------------------------------------------------------------------------

% List of supporting notes  (leave as null \AtlasCoverSupportingNote{} if you want to skip this option)
%\AtlasCoverSupportingNote{Short title note 1}{https://cds.cern.ch/record/XXXXXXX}
% \AtlasCoverSupportingNote{Short title note 2}{https://cds.cern.ch/record/YYYYYYY}
%
% OR (the 2nd option is deprecated, especially for CONF and PUB notes)
%
% Supporting material TWiki page  (leave as null \AtlasCoverTwikiURL{} if you want to skip this option)
%\AtlasCoverTwikiURL{https://twiki.cern.ch/twiki/bin/view/Atlas/WebHome}

% Comment deadline
% \AtlasCoverCommentsDeadline{DD Month 2014}

% Analysis team members - contact editors should no longer be specified
% as there is a generic email list name for the editors
% \AtlasCoverAnalysisTeam{Peter Analyser, Susan Editor1, Jenny Editor2, Alphonse Physicien}

% Editorial Board Members - indicate the Chair by a (chair) after his/her name
% Give either all members at once (then they appear on one line), or separately
% \AtlasCoverEdBoardMember{EdBoard~Chair~(chair), EB~Member~1, EB~Member~2, EB~Member~3}
% \AtlasCoverEdBoardMember{EdBoard~Chair~(chair)}
% \AtlasCoverEdBoardMember{EB~Member~1}
% \AtlasCoverEdBoardMember{EB~Member~2}
% \AtlasCoverEdBoardMember{EB~Member~3}

% A PUB note has readers and not an EdBoard -- give their names here (one line or several entries)
% \AtlasCoverReaderMember{Reader~1, Reader~2}
% \AtlasCoverReaderMember{Reader~1}
% \AtlasCoverEdBoardMember{Reader~2}

% Editors egroup
% \AtlasCoverEgroupEditors{atlas-GROUP-2014-XX-editors@cern.ch}

% EdBoard egroup
% \AtlasCoverEgroupEdBoard{atlas-GROUP-2014-XX-editorial-board@cern.ch}

