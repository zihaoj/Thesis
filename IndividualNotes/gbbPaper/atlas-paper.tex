%-------------------------------------------------------------------------------
% This file provides a skeleton ATLAS paper.
%-------------------------------------------------------------------------------
% \pdfoutput=1
% The \pdfoutput command is needed by arXiv/JHEP/JINST to ensure use of pdflatex.
% It should be included in the first 5 lines of the file.
% \pdfinclusioncopyfonts=1
% This command may be needed in order to get \ell in PDF plots to appear. Found in
% https://tex.stackexchange.com/questions/322010/pdflatex-glyph-undefined-symbols-disappear-from-included-pdf
%-------------------------------------------------------------------------------
% Specify where ATLAS LaTeX style files can be found.
\newcommand*{\ATLASLATEXPATH}{latex/}
% Use this variant if the files are in a central location, e.g. $HOME/texmf.
% \newcommand*{\ATLASLATEXPATH}{}
%-------------------------------------------------------------------------------
\documentclass[PAPER, atlasdraft=true, texlive=2016, UKenglish,coverpage]{\ATLASLATEXPATH atlasdoc}
% The language of the document must be set: usually UKenglish or USenglish.
% british and american also work!
% Commonly used options:
%  atlasdraft=true|false This document is an ATLAS draft.
%  texlive=YYYY          Specify TeX Live version (2016 is default).
%  coverpage             Create ATLAS draft cover page for collaboration circulation.
%                        See atlas-draft-cover.tex for a list of variables that should be defined.
%  cernpreprint          Create front page for a CERN preprint.
%                        See atlas-preprint-cover.tex for a list of variables that should be defined.
%  NOTE                  The document is an ATLAS note (draft).
%  PAPER                 The document is an ATLAS paper (draft).
%  CONF                  The document is a CONF note (draft).
%  PUB                   The document is a PUB note (draft).
%  BOOK                  The document is of book form, like an LOI or TDR (draft)
%  txfonts=true|false    Use txfonts rather than the default newtx
%  paper=a4|letter       Set paper size to A4 (default) or letter.

%-------------------------------------------------------------------------------
% Extra packages:
\usepackage{\ATLASLATEXPATH atlaspackage}
\usepackage{comment}
\usepackage{booktabs}
% Commonly used options:
%  biblatex=true|false   Use biblatex (default) or bibtex for the bibliography.
%  backend=bibtex        Use the bibtex backend rather than biber.
%  subfigure|subfig|subcaption  to use one of these packages for figures in figures.
%  minimal               Minimal set of packages.
%  default               Standard set of packages.
%  full                  Full set of packages.
%-------------------------------------------------------------------------------
% Style file with biblatex options for ATLAS documents.
\usepackage{\ATLASLATEXPATH atlasbiblatex}

% Useful macros
\usepackage{\ATLASLATEXPATH atlasphysics}
% See doc/atlas_physics.pdf for a list of the defined symbols.
% Default options are:
%   true:  journal, misc, particle, unit, xref
%   false: BSM, heppparticle, hepprocess, hion, jetetmiss, math, process, other, texmf
% See the package for details on the options.

% Files with references for use with biblatex.
% Note that biber gives an error if it finds empty bib files.
% \addbibresource{atlas-document.bib}
\addbibresource{bib/ATLAS.bib}
\addbibresource{bib/CMS.bib}
\addbibresource{bib/ConfNotes.bib}
\addbibresource{bib/PubNotes.bib}
\addbibresource{int.bib}

\usepackage{tikz}
\usetikzlibrary{decorations.markings}
\usetikzlibrary{snakes}
\tikzset{
    fermion/.style={draw=black, postaction={decorate},
        decoration={markings,mark=at position .55 with {\arrow[draw=black]{>}}}},
     gluon/.style={decorate, draw=black,
        decoration={coil,amplitude=3pt, segment length=5pt}},
}

% Paths for figures - do not forget the / at the end of the directory name.
\graphicspath{{logos/}{figures/}}

% Add you own definitions here (file atlas-document-defs.sty).

\usepackage{atlas-paper-defs}

%-------------------------------------------------------------------------------
% Generic document information
%-------------------------------------------------------------------------------

% Title, abstract and document 
%-------------------------------------------------------------------------------
% This file contains the title, author and abstract.
% It also contains all relevant document numbers used by the different cover pages.
%-------------------------------------------------------------------------------

% Title
\AtlasTitle{Measuring the properties of $g\rightarrow b\bar{b}$ at small opening angles in $pp$ collisions with the ATLAS detector at $\sqrt{s}=13$ TeV}

% Draft version:
% Should be 1.0 for the first circulation, and 2.0 for the second circulation.
% If given, adds draft version on front page, a 'DRAFT' box on top of each other page, 
% and line numbers.
% Comment or remove in final version.
\AtlasVersion{1.5}

% Abstract - % directly after { is important for correct indentation
\AtlasAbstract{%
The fragmentation of high energy gluons at small opening angles is largely unconstrained by present measurements.  Gluon splitting to $b$-quark pairs is a unique probe into the properties of gluon fragmentation because identified $b$-tagged jets provide a proxy for the quark daughters of the initial gluon.  In this study, key differential distributions related to the $g\rightarrow b\bar{b}$ process are measured using $33$ fb$^{-1}$ of $\sqrt{s}=13$ TeV $pp$ collision data recorded by the ATLAS experiment at the LHC in 2016.  Jets constructed from charged-particle tracks, clustered with the jet anti-$k_t$ algorithm with radius parameter $R = 0.2$, are used to probe angular scales below the $R=0.4$ jet radius.  The observables are unfolded to particle level in order to facilitate direct comparisons with predictions from present and future simulations.  Multiple significant differences are observed between the data and parton shower Monte Carlo predictions, providing input to improve these predictions of the main source of background events in analyses involving boosted Higgs bosons decaying to $b$-quarks.
}

% Author - this does not work with revtex (add it after \begin{document})
% This has to be commented out for TDR etc.
% \author{The ATLAS Collaboration}

% ATLAS reference code, to help ATLAS members to locate the paper
\AtlasRefCode{STDM-2017-17}

% CERN preprint number
% \PreprintIdNumber{CERN-EP-2017-XX}

% ATLAS date - arXiv submission; usually filled in by the Physics Office
% \AtlasDate{\today}

% ATLAS heading - heading at top of title page. Set for TDR etc.
% \AtlasHeading{ATLAS ABC TDR}

% arXiv identifier
% \arXivId{14XX.YYYY}

% HepData record
% \HepDataRecord{ZZZZZZZZ}

% Submission journal and final reference
 \AtlasJournal{Phys. Rev. D.}
% \AtlasJournalRef{\PLB 789 (2017) 123}
% \AtlasDOI{}

%-------------------------------------------------------------------------------
% The following information is needed for the cover page. The commands are only defined
% if you use the coverpage option in atlasdoc or use the atlascover package
%-------------------------------------------------------------------------------

% List of supporting notes  (leave as null \AtlasCoverSupportingNote{} if you want to skip this option)
% \AtlasCoverSupportingNote{Short title note 1}{https://cds.cern.ch/record/XXXXXXX}
 \AtlasCoverSupportingNote{Support note}{https://cds.cern.ch/record/2289245}
%
% OR (the 2nd option is deprecated, especially for CONF and PUB notes)
%
% Supporting material TWiki page  (leave as null \AtlasCoverTwikiURL{} if you want to skip this option)
% \AtlasCoverTwikiURL{https://twiki.cern.ch/twiki/bin/view/Atlas/WebHome}

% Comment deadline
 \AtlasCoverCommentsDeadline{September 14, 2018}

% Analysis team members - contact editors should no longer be specified
% as there is a generic email list name for the editors
 \AtlasCoverAnalysisTeam{Zihao Jiang, Benjamin Nachman, Lauren Tompkins}

% Editorial Board Members - indicate the Chair by a (chair) after his/her name
% Give either all members at once (then they appear on one line), or separately
% \AtlasCoverEdBoardMember{EdBoard~Chair~(chair), EB~Member~1, EB~Member~2, EB~Member~3}
% \AtlasCoverEdBoardMember{EdBoard~Chair~(chair)}
 \AtlasCoverEdBoardMember{John~Alison~(previous chair; now in CMS)}
 \AtlasCoverEdBoardMember{Andy~Buckley~(chair)}
 \AtlasCoverEdBoardMember{Ian~Shipsey}

% A PUB note has readers and not an EdBoard -- give their names here (one line or several entries)
% \AtlasCoverReaderMember{Reader~1, Reader~2}
% \AtlasCoverReaderMember{Reader~1}
% \AtlasCoverEdBoardMember{Reader~2}

% Editors egroup
 \AtlasCoverEgroupEditors{atlas-stdm-2017-17-editors@cern.ch}

% EdBoard egroup
 \AtlasCoverEgroupEdBoard{atlas-stdm-2017-17-editorial-board@cern.ch}


% Author and title for the PDF file
\hypersetup{pdftitle={ATLAS document},pdfauthor={The ATLAS Collaboration}}

%-------------------------------------------------------------------------------
% Content
%-------------------------------------------------------------------------------
\begin{document}

\maketitle

\tableofcontents


%-------------------------------------------------------------------------------
\section{Introduction}
\label{sec:intro}
One critical research area at the Large Hadron Collider (LHC) is the search for highly Lorentz boosted Higgs bosons produced from the Standard Model (SM)~\cite{Sirunyan:2017dgc} or from beyond the SM processes~\cite{Aaboud:2017ecz,Aaboud:2017ahz,Aaboud:2017yqz,Aaboud:2016xco,Khachatryan:2016cfa,Sirunyan:2017nrt}.  As the branching ratio for the Higgs to decay into bottom quark pairs dominates the total decay rate, the boosted $H\rightarrow b\bar{b}$ channel is often the most sensitive.  Algorithms for identifying jets resulting from bottom quark fragmentation are very powerful, so the main background for boosted Higgs boson searches contains real $b$-quarks.  The main contribution to this background is gluon splitting to $b$-$\bar{b}$ pairs at small opening angles since the angle between the $b$ quarks in $H\rightarrow b\bar{b}$ scales with $m_H/p_H$.  The $g\rightarrow b\bar{b}$ process also contributes to many other important SM measurements and searches, providing a source of additional real $b$-quark jets that can fake a signal where $b$ quarks originate from elsewhere.  


The modeling of $g\rightarrow b\bar{b}$ fragmentation is complex and interesting in its own right.  The significant mass of the $b$-quark modifies the usual quantum chromodynamic (QCD) splitting functions.  Trijet events from LEP~\cite{Barate:1998vs,Abbiendi:2000zt,Abreu:1999qh} and SLD~\cite{Abe:2001csa} provide valuable information on the rate of $g\rightarrow b\bar{b}$, but do not constrain the differential properties of the fragmentation in the small opening angle regime.  Previous measurements that include the the $b\bar{b}$ final state at the S$p\bar{p}$S, Tevatron, and LHC using inclusive~\cite{Albajar:1993be,Albajar:1990zu,Aaboud:2017vqt,Khachatryan:2011wq,Chatrchyan:2012hw,Aad:2011rr,Abbott:1999se,Aaltonen:2007zza,Abbott:1999wu,Abachi:1994kj,Khachatryan:2011wq,Aaij:2017pvu,Aaij:2010gn,Aaij:2013noa,Aaij:2012jd,Aaij:2010gn,Aad:2012jga,Aad:2011sp,Aad:2015duc,Khachatryan:2010yr,Acosta:2004yw,ATLAS:2013cia}, multijet~\cite{Aaboud:2016jed,ATLAS:2011ac,Chatrchyan:2012dk}, and associated production~\cite{Aad:2013vka,Aad:2014dvb,Chatrchyan:2014dha,Abazov:2014fka,Abazov:2013uza,Abazov:2015aha,Aaltonen:2013ama,Aaltonen:2009qi,Aaltonen:2008mt,Chatrchyan:2013zja,Chatrchyan:2013zja} topologies have focused on well-separated quark pairs and were limited in their kinematic reach due in part to limited dataset sizes and low $\sqrt{s}$.

The high transverse momentum and low angular separation regime for $g\rightarrow b\bar{b}$ can be probed at the LHC using large-radius jets with $b$-tagged subjets. This topology has been used for calibrating $b$-tagging in dense environments~\cite{ATLAS-CONF-2016-039,ATLAS-CONF-2016-002,CMS:2013vea} and has been studied phenomenologically~\cite{Anderle:2017qwx,Ilten:2017rbd}.  This measurement builds on these studies to perform the first differential cross-section measurement of the $g\rightarrow b\bar{b}$ at high transverse momentum inside jets.  Small-radius jets built from charged particle tracks are used as proxies for the $b$-quark fragmentation and allow for a precision probe of the small opening angle regime.


%-------------------------------------------------------------------------------

%-------------------------------------------------------------------------------
\section{ATLAS detector}
\label{sec:detector}

The ATLAS detector~\cite{ref:AtlasDet} is a general-purpose particle detector used to investigate 
a broad range of physics processes. It includes inner tracking devices surrounded by a 2.3 m diameter superconducting 
solenoid, electromagnetic and hadronic calorimeters and a muon spectrometer with a toroidal magnetic field. 
The inner detector consists of a high-granularity silicon pixel detector, including the insertable 
B-layer~\cite{ref:IBL} installed after Run 1 of the LHC, a silicon strip detector, and a straw-tube tracker; 
it is situated inside a 2 T axial magnetic field and provides precision tracking of charged particles with 
pseudorapidity $|\eta| <$ 2.5\footnote{ATLAS uses a right-handed coordinate system with its origin 
at the nominal interaction 
point (IP) in the centre of the detector and the $z$-axis along the beam pipe. The $x$-axis points 
from the IP to the centre of the LHC ring, and the $y$-axis points upward. Cylindrical coordinates 
$(r,\phi)$ are used in the transverse plane, $\phi$ being the azimuthal angle around the $z$-axis. 
The pseudorapidity is defined in terms of the polar angle $\theta$ as $\eta=-\ln\tan(\theta/2)$. 
The rapidity is also defined relative to the beam axis as $y = \frac{1}{2}\ln \Big( \frac{E + p_z}{E - p_z} \Big) $. }. 
The calorimeter system consists 
of finely segmented sampling calorimeters using lead/liquid-argon for the 
detection of electromagnetic (EM)
showers up to $|\eta| <$ 3.2, and copper or tungsten/liquid-argon for 
hadronic showers for 1.5 $< |\eta| <$ 4.9.  In the central 
region ($|\eta| <$ 1.7), an steel/scintillator hadronic calorimeter is used.
Outside the calorimeters, the muon system incorporates multiple layers of
trigger and tracking chambers within a magnetic field produced by a system of superconducting toroids,
enabling an independent, precise measurement of muon track momenta for $|\eta| <$ 2.7.
The ATLAS detector has a two-level 
trigger system to select events for offline analysis~\cite{ATL-DAQ-PUB-2016-001}.



%-------------------------------------------------------------------------------
\section{Datasets}
\label{sec:data}
\input{sections/datasets}
%-------------------------------------------------------------------------------

%-------------------------------------------------------------------------------
\section{Object and Event Selection}
\label{sec:obj}
\label{sec:gbb-obj}

\subsubsection{Primary Vertex and Track Selection}
The primary hard scattering vertex for the event is chosen as the vertex with the highest $\sum p_{T}^{2}$ where the sum runs over the tracks associated to the vertex. Tracks which are used to determine the primary vertex need to pass the following selection:
\begin{enumerate}
    \item $p_T>$ 0.5 GeV
    \item Number of Pixel hits greater than 1
    \item Number of SCT hits greater than 6
    \item $|d_0|<1$mm, with respect to primary vertex
    \item $|z_0\sin(\theta)|<1$mm, with respect to primary vertex
\end{enumerate}

\subsubsection{Jet Selection}
In this study we use $R$=1.0 calorimeter jets and $R=0.2$ track jets. Calorimeter jets are clustered in $y-\phi$ space using the anti-$k_t$~\cite{Cacciari:2008gp} algorithm with $R=1.0$ reconstructed from topological calorimeter clusters~\cite{TopoClusters} using the local cluster weighting (LCW) algorithm \cite{EndcapTBelectronPion2002} and calibrated to account for the detector response. Track jets are reconstructed clustering inner detector tracks using the anti-$k_t$ algorithm and are required to have at least two tracks.

\noindent The $R = 1.0$ calorimeter jets are groomed using the trimming procedure~\cite{Krohn:2009th}, whereupon the $k_{t}$ subjets ~\cite{Cacciari200657} with $R=0.3$ are discarded if the fractional $p_{T}$ of the subject relative to the whole $R=1.0$ jet satisfies $f_\text{ cut}<0.05$.  

\noindent The $R=0.2$ track jets are subject to the following selection, 
\begin{enumerate}
	\item $p_T>$10 GeV
	\item $|\eta|<2.5$	
	\item The track jet originates from the primary vertex (OriginIndex==0)
\end{enumerate}

\subsubsection{$b$-tagging}

Jets are identified as $b$-jets using the multivariate discriminant $MV2c10$ \cite{btag} which includes impact parameter and secondary vertex information as inputs.  The chosen $MV2c10$ working point corresponds to an average $b$-tagging efficiency of 70\% for $b$-jets in simulated $t\bar{t}$ events.  

\subsubsection{Ghost Association of Jets}

We adopt a robust matching algorithm called ghost association~\cite{area} to associate $R=0.2$ track jets to $R=1.0$ jets. With this method, the 4-vector of the physics object is added to the inputs of the jet clustering algorithm but the object 4-vector has the $p_T$ set to an infinitesimal amount, hence called a ghost.  Jet clustering is then performed. This way, the ghost does not alter the clustering history, but the 4-vectors retain the original object direction and are clustered into a jet if the ghost 4-vector points within the active area of the jet.

\subsubsection{Flavor Labeling of Jets in Simulation}

The flavor content of the track jet is determined by ghost matching truth particles (weakly decaying $B$-hadrons and $c$-hadrons) to the track jet. For each jet, if a $B$-hadron is found to be associated to the jet, then the jet is labeled as a $b$-jet.  If there are no $B$-hadrons but a $c$-hadron is found to be associated to the jet, then the jet is labeled as a $c$-jet. Otherwise, the jet is labeled as a light-flavored jet. 

\subsubsection{Truth Jets}

All final state truth particles (ignoring truth pileup) with mean lifetimes longer than 30 ps, except muons and neutrinos, are used as input to the clustering of truth jets. 

%-------------------------------------------------------------------------------

%-------------------------------------------------------------------------------
\section{Observables}
\label{sec:obs}
\input{sections/observables}
%-------------------------------------------------------------------------------

%-------------------------------------------------------------------------------
\section{Background Estimation}
\label{sec:analysis}
\label{sec:gbb-bkgsub}
\subsection{Flavor Fractions}
Post \btagging, a large fraction of events are backgrounds as shown in Table~\ref{tab:posttaggingflav}. To unfold the data, subtraction of the remnant background is necessary. It is a known problem that the nominal MC flavor fractions could deviate from the true values in data (see e.g.~\cite{ATLAS-CONF-2016-002}). To have good control over the flavor fractions, we seek to estimate the flavor fractions in bins of each observable by fitting the distributions of some variables, which may not be powerful enough to tag individual jets, are sensitive to the numbers of jets of different flavors. Once we fit the flavor composition, we apply $b$-tagging and examine the remaining difference between MC and data, and check if the difference lies within the uncertainty bands of the $b$-tagging systematic uncertainties. 

\begin{table}[htbp]
\centering
\begin{tabular}{|l|l|l|l|l|l|l|l|l|l|}
\hline
Flavor Combination & BB      & BC     & BL     & CB     & CC     & CL     & LB     & LC     & LL     \\ \hline
Flavor Fraction    & 19.01\% & 2.20\% & 34.68\% & 0.39\% & 5.73\% & 17.99\% & 0.37\% & 0.86\% & 18.76\% \\ \hline
\end{tabular}
\caption{Post $b$-tagging flavor composition in MC. The first letter of the flavor combination is the flavor of the leading jet. The second letter of the combination is the flavor of the sub-leading jet. }
\label{tab:posttaggingflav}
\end{table}


\subsection{$s_{d_{0}}$ as Discriminant Variable }
\label{sec:gbb-sd0}

The long decay length of heavy flavor hadrons make the signed significance of impact parameter $s_{d_{0}}$ of tracks associated to a jet a good discriminating variable. The $s_{d_{0}}$ is defined as 
\begin{equation}
s_{d_{0}} = \frac{d_0}{\sigma(d_0)} \cdot s_{j}
\end{equation}
where $d_{0}$ is the track transverse impact parameter, $\sigma(d_{0})$ is the uncertainty on the $d_0$ measurement, and $s_{j}$ is the sign of $d_{0}$ with respect to the jet axis. The variable $s_j$ is defined as
\begin{equation}
s_{j} = \textrm{sign}\left\{\ \sin\left(\arctan \left( \frac{p_{j,\ y}}{p_{j,\ x}}\right) - \phi_t\right) \cdot d_{0} \ \right\}
\end{equation}
 where $p_{j,\ x}$ and $p_{j,\ y}$ are the $x$ and $y$ components of the jet moments, respectively, and $\phi_{t}$ is the azimuthal angle of the track.
 
For a given track jet, by construction we have at least two tracks as constituents of the jet. %(number of track constituents distributions are shown in Fig.\ref{fig:gbb-ntrk}).
We take the sub-leading $s_{d_{0}}$ of the track (leading means highest value of $|s_{d_{0}}|$, sub-leading is second largest, etc.) as \subsdzero and build templates of different flavors using this variable to fit and derive the flavor fractions from data.   The second highest $s_{d_{0}}$ is used because it is less likely the result of mis-measured track parameters for non-$b$-jets.  For $b$-jets, the distributions of leading and sub-leading are very similar\footnote{See e.g. \href{https://indico.cern.ch/event/642521/contributions/2607425/subcontributions/230337/attachments/1517581/2369090/gbb_09_03_ZihaoJiang_Rel21Scrutiny.pdf}{these slides}.}. 


%\begin{figure}[htbp]
%  \centering
% \includegraphics[width=0.48\textwidth]{figures/gbb/Leading_trkjet_ntrk_PostTag.pdf}
% \includegraphics[width=0.48\textwidth]{figures/gbb/SubLeading_trkjet_ntrk_PostTag.pdf}\\
%\caption{The number of track distribution for leading (left) and sub-leading (right) track jets in MC prediction}
%  \label{fig:gbb-ntrk}
%\end{figure}


%%%%%%%%%%%%%%%%%%%%%%%%%%%%%%%%%%%%%%%%%%%%%%%%%

\subsection{Binned Maximum Likelihood Estimation}

An $m$ flavor component template fit ($m = 3$ in the final fit we perform) is carried out by maximizing a binned likelihood function defined as $\mathcal{L} = \prod _{i=1} ^n p(y_i| \vec{\theta})$, where $\vec{\theta} \in \mathbb{R}^m$ denotes the number of events from each flavor component, $n$ is the number of bins we are fitting and $y_i$ is the number of data events in bin $i$. The assumption of $p(y_i|\vec{\theta})$ is Poisson distribution: 
\begin{equation}
p(y_i|\vec{\theta}) = \frac{\exp{(-\vec{\theta} \vec{\cdot x_i)}} {(\vec{\theta}} \vec{\cdot x_i})^{y_i}}{y_i !}, 
\end{equation}
where $\vec{x_i}=(x_{i1}...,x_{ij}...,x_{im}) \in \mathbb{R}^m$ is vector, and each component denotes the $i^{th}$ bin value in the normalized templates for the $j^{th}$ flavor ($b$, $c$, light and etc.).   We determine the flavor fractions by finding the $\vec{\theta}$ that maximizes $\mathcal{L}$. The fit is carried out by the MIGRAD function of the MINUIT package. Since we have limited statistics, the MC statistics are profiled in the likelihood fits through Gaussian priors, i.e. the bin counts in MC template are allowed to float with a Gaussian penalization term introduced in the likelihood function.


%%%%%%%%%%%%%%%%%%%%%%%%%%%%%%%
\subsection{\subsdzero Templates }

The flavor contents of the two $R=0.2$ jets in a $R=1.0$ jet are correlated. 
Tagging one of the $b$ jets increases the probability of the other track jet 
being a $b$ jet. Therefore, we should not treat the flavor 
contents of the two jets independently and thus must take into account both jets 
flavors simultaneously when fitting. On the other hand, given the flavors of the 
two $R= 0.2$ jets, their \subsdzero values are not correlated, as shown in Table \ref{tab:sd0cor}. 
Therefore, we can derive the flavor fractions by fitting simultaneously the 
1-dimensional \subsdzero distributions of the leading and sub-leading jets
 using templates derived from all of the possible 2-jet flavor pair 
combinations. This is essentially a Naive Bayes approximation, i.e. assuming the joint 
2-D probability density is the product of the two 1-D probability density
 $p(\subsdzero(j1), \subsdzero(j2)) = p(\subsdzero(j1))p(\subsdzero(j2))$.

\begin{table}[htbp]
\centering
\begin{tabular}{|r|r|r|r|}
\hline
\centering
Flavor Combination & \subsdzero Correlation Factor\\
\hline
BB & 0.7\% \\
BL & 0.4\% \\
BC & -2.2\% \\
LB & 3.9\%\\
LL & 0.3\%\\
LC & 0.8\%\\
CB & -0.3\% \\
CL & 0.5\% \\
CC & -4.1\% \\

\hline
\end{tabular}%
\caption{Correlation factor between the \subsdzero values of the two R = 0.2 jets.}
\label{tab:sd0cor}
\end{table}%

There are in total nine different ordered flavor pairs: BB, BL, BC, CB, CL, CC, 
LB, LC and LL such that the first flavor is that of the leading track jet and 
the second is that of the other track jet. Many of these components are very small 
for us to fit their contributions correctly as show in Table~\ref{tab:posttaggingflav}. 
Therefore, starting with BL, CL templates, we merge other templates to these templates
to form B-like and C/L-like templates. The metric we use for determining the similarity 
between two distributions is: $S=\frac{1}{2}\int \frac{(p_1(x)-p_2(x))^2}{p_1(x)+p_2(x)}$, 
where $p_1(x)$ and $p_2(x)$ are two different distributions. 
Based on calculation, we merge the BC template with the BL 
template, and other components into CL as seen in 
Table \ref{tab:overlap-unmerged}. The merged templates are 
presented in Figure \ref{fig:gbb-templates} and the separation powers estimated by the same metric are 
presented in Table \ref{tab:overlap-merged}. 
We show the full set of templates in bins of the observables we are interested in 
measuring in Appendix~\ref{sec:gbb-app-sd0templates}. 

It is noteworthy that the statistics of our Monte Carlo samples is low.
Presumably, a double $b$-tagging strategy would be preferable as it could filter 
out most of the background processes.
However, double $b$-tagging strategy would also yield background templates with large 
statistical fluctuations. In Fig. \ref{fig:gbb-template-leadtight-medium} the comparison 
of the templates derived from single $b$-tagging and double $b$-tagging (at 70\% $b$ 
efficiency for both jets) in one bin of $\Delta R$ is shown. The considerable statistical uncertainties
of double $b$-tagging strategy lead us to use single $b$-tagging strategy instead. 


\begin{table}[htpb]
\centering
\begin{tabular}{|l|l|l|l|l|}
\hline
   & BL   & CL    & BB   \\ \hline
BC & 0.02 & 0.08  & 0.11 \\ \hline
CC & 0.09 & 0.03  & 0.14 \\ \hline
LC & 0.07 & 0.05  & 0.12 \\ \hline
LB & 0.17 & 0.12  & 0.05 \\ \hline
CB & 0.19 & 0.14  & 0.08 \\ \hline
LL & 0.02 & 0.01  & 0.17 \\ \hline
BB & 0.16 & 0.21  & 0    \\ \hline

\end{tabular}
\caption{Templates similarity calculated using the metric $S=\frac{1}{2}\int \frac{(p_1(x)-p_2(x))^2}{p_1(x)+p_2(x)}$ for un-merged templates. Given two distributions $p_1(x)$ and $p_2(x)$, the metric returns 0 if they are exactly the same and returns 1 if there is no overlap between them.}
\label{tab:overlap-unmerged}
\end{table}



\begin{table}[htpb]
\centering
\begin{tabular}{|l|l|l|l|l|}
\hline
    & B    & L+C    & BB   \\ \hline
B   & 0    & 0.04  & 0.16 \\ \hline
L+C & 0.04 & 0     & 0.17 \\ \hline
BB  & 0.16 & 0.17  & 0    \\ \hline

\end{tabular}
\caption{Templates similarity calculated using the metric $S=\frac{1}{2}\int \frac{(p_1(x)-p_2(x))^2}{p_1(x)+p_2(x)}$ for merged templates. Given two distributions $p_1(x)$ and $p_2(x)$, the metric returns 0 if they are exactly the same and returns 1 if there is no overlap between them.}
\label{tab:overlap-merged}
\end{table}


\begin{figure}[htbp]
  \centering
 \includegraphics[width=0.48\textwidth]{figures/gbb/Sub_Sd0_Fits/Canv_FitTemplate_Inclusive_x.pdf}
 \includegraphics[width=0.48\textwidth]{figures/gbb/Sub_Sd0_Fits/Canv_FitTemplate_Inclusive_y.pdf}\\
\caption{The merged templates (inclusive) for leading (left) and sub-leading (right) track jets.}
  \label{fig:gbb-templates}
\end{figure}


\begin{figure}[htbp]
  \centering
 \includegraphics[width=0.48\textwidth]{figures/gbb/Canv_FitTemplate_medium.pdf}
 \includegraphics[width=0.48\textwidth]{figures/gbb/Sub_Sd0_Fits/Canv_FitTemplate_02-DeltaR-025_LpT_INF_SpT_INF_x.pdf}\\
\caption{The merged templates for leading track jets of events with $0.2<\Delta R<0.25$ for double $b$-tagging at 70\% eff WP (left) and single $b$-tagging (right). Post double $b$-tagging the statistics of background templates is too low.}
 \label{fig:gbb-template-leadtight-medium}
\end{figure}


%%%%%%%%%%%%%%%%%%%%%%%%%%%%%%%
\subsection{Systematic Uncertainties}
\label{sec:gbb-sub_systematics}

Separate fits are performed for all the $\pm 1\sigma$ variations of detector systematics listed in Section~\ref{sec:gbb-systs:exp}. In addition, a number of systematic uncertainties for the flavor fraction fits are taken into account for the background determination of the nominal fit:
\begin{enumerate}
  \item \textbf{Data and MC statistical uncertainties:} The fit statistical uncertainties are taken directly from the errors from the MINUIT fitting package and propagated to unfolding.
  \item \textbf{Fit range:} The nominal flavor fraction fit is performed in \subsdzero$\in[-40, 70]$. To avoid having the potential tail mis-modeling of \subsdzero affects the result, fits are separately performed excluding the tails on the left $[-40, -30]$ and right side $[60, 70]$ of the \subsdzero distributions. The $b\bar b$ fitted fraction differences between the left/right-excluded fits and the nominal fit are propagated to unfolding machinery as one source of uncertainty. 
  \item \textbf{Template merging scheme:} The merge of small components to form aggregated templates essentially fixes the relative fractions of these components. The systematic uncertainties caused by the merging scheme is estimated by varying the contribution of each merged component up and down by a factor of two. The template fits are performed for each of these variations. In total $8\times 2 =16$ such variations are propagated to unfolding.
  \item \textbf{Alternative discriminants:} The robustness of the choice using \subsdzero as our fitting templates is checked against another choice using the leading and third-leading $s_{d0}$ value of the track jets, which we define as \sdzero and \subsubsdzero respectively. Note that fitting with these two different variables serves only as a closure check for different background modeling as shown in Appendices~\ref{sec:gbb-app-sd0} and \ref{sec:gbb-app-subsubsd0}. The difference of fitted flavor fractions are not propagated as a systematic uncertainty associated with unfolding.
  \item \textbf{Fitting with \pt parameterized templates:} The nominal flavor fraction fit with inclusive \subsdzero templates are checked against fitting templates parameterized with track jets \pt. For each bin of observable, the flavor fraction fit is performed in four bins of leading (J1) and sub-leading (J2) track jet \pt: $(p_T(J1), p_T(J2))$. The four bins are $(p_T(J1)<200$~$\GeV, p_T(J2)<80$~$\GeV)$, $(p_T(J1)>200$~$\GeV, p_T(J2)<80$~$\GeV)$, $(p_T(J1)<200$~$\GeV, p_T(J2)>80$~$\GeV)$ and $(p_T(J1)>200$~~$\GeV, p_T(J2)>80$~$\GeV)$. The cross check is performed to check closure. The difference of fitted flavor fractions are not propagated as a systematic uncertainty associated with unfolding.
  \item \textbf{Kinematic reweighting:} After the event selection, there are mild disagreements between data and MC of jet kinematic properties. The impact of the disagreement on the final results is explored by reweighting the MC, event-by-event, by the ratio of the two dimensional leading and sub-leading track jets \pt and $\eta$ distributions between MC and data post \btagging as shown in Fig.~\ref{fig:gbb-reweightmap} \footnote {Note that $b$-tagging scale factors are applied prior to re-weighting.}. The mis-modeling could arise from the overall di-jet cross section, $R=1.0$ ghost matching efficiency and $b$-tagging efficiency as a function of \pt in the particular event topology. The data and MC comparison for the \pt of the $R=1.0$ and track jets are shown after applying $b$-tagging with and without kinematics reweighting in Fig.~\ref{fig:gbb-pT_largeR},\ref{fig:gbb-pT_leadtrkjets},\ref{fig:gbb-pT_subtrkjets}. We do not find the reweighting affects the flavor fit in any significant way and hence do not apply the reweighting for the nominal results and the difference of fitted flavor fractions are not propagated as a systematic uncertainty associated with unfolding.

%\ref{fig:gbb-eta_largeR} \ref{fig:gbb-eta_leadtrkjets},\ref{fig:gbb-eta_subtrkjets}. The effects of applying kinematics reweighting is checked. A cross check is performed fitting data with templates derived from reweighted sample.

\end{enumerate}

\begin{figure}[htbp]
  \centering
 \includegraphics[width=0.6\textwidth]{figures/gbb/pTReweightMap.pdf}
\caption{The re-weighting factor applied to MC as a function of leading and sub-leading track jet \pt.}
  \label{fig:gbb-reweightmap}
\end{figure}


\begin{figure}[htbp]
  \centering
 %\includegraphics[width=0.38\textwidth]{figures/gbb/LargeRJet_pT_NoReweight.pdf}
 \includegraphics[width=0.38\textwidth]{figures/gbb/LargeRJet_pT_PreReweight.pdf}
 \includegraphics[width=0.38\textwidth]{figures/gbb/LargeRJet_pT_Reweight.pdf}
 \includegraphics[width=0.38\textwidth]{figures/gbb/LargeRJet_eta_PreReweight.pdf}
 \includegraphics[width=0.38\textwidth]{figures/gbb/LargeRJet_eta_Reweight.pdf}
\caption{Data/MC comparison of $R=1.0$ jet \pt (top) and $\eta$ (bottom) post $b$-tagging without kinematic reweighting (left) and post $b$-tagging with kinematic reweighting (right).}% The label of the $R=1.0$ jet flavor content ``XY'' denotes the leading and sub-leading track jet flavor. For example, the flavors of the leading and sub-leading track jet of a ``BL'' $R=1.0$ jet are `B' and `Light' respectively.}
  \label{fig:gbb-pT_largeR}
\end{figure}


\begin{figure}[htbp]
  \centering
  %\includegraphics[width=0.38\textwidth]{figures/gbb/LeadTrkJet_pT_NoReweight.pdf}
 \includegraphics[width=0.38\textwidth]{figures/gbb/LeadTrkJet_pT_PreReweight.pdf}
 \includegraphics[width=0.38\textwidth]{figures/gbb/LeadTrkJet_pT_Reweight.pdf}
 \includegraphics[width=0.38\textwidth]{figures/gbb/LeadTrkJet_eta_PreReweight.pdf}
 \includegraphics[width=0.38\textwidth]{figures/gbb/LeadTrkJet_eta_Reweight.pdf}
\caption{Data/MC comparison of leading track jets \pt (top) and $\eta$ (bottom) post $b$-tagging without kinematic reweighting (left) and post $b$-tagging with kinematic reweighting (right).} % The label of the $R=1.0$ jet flavor content ``XY'' denotes the leading and sub-leading track jet flavor. For example, the flavors of the leading and sub-leading track jet of a ``BL'' $R=1.0$ jet are `B' and `Light' respectively.}
  \label{fig:gbb-pT_leadtrkjets}
\end{figure}


\begin{figure}[htbp]
  \centering
%\includegraphics[width=0.38\textwidth]{figures/gbb/SubLeadTrkJet_pT_NoReweight.pdf}
\includegraphics[width=0.38\textwidth]{figures/gbb/SubLeadTrkJet_pT_PreReweight.pdf}
 \includegraphics[width=0.38\textwidth]{figures/gbb/SubLeadTrkJet_pT_Reweight.pdf}
\includegraphics[width=0.38\textwidth]{figures/gbb/SubLeadTrkJet_eta_PreReweight.pdf}
 \includegraphics[width=0.38\textwidth]{figures/gbb/SubLeadTrkJet_eta_Reweight.pdf}
\caption{Data/MC comparison of sub-leading track jets \pt (top) and $\eta$ (bottom) post $b$-tagging without kinematic reweighting (left) and post $b$-tagging with kinematic reweighting (right).}% The label of the $R=1.0$ jet flavor content ``XY'' denotes the leading and sub-leading track jet flavor. For example, the flavors of the leading and sub-leading track jet of a ``BL'' $R=1.0$ jet are `B' and `Light' respectively.}
  \label{fig:gbb-pT_subtrkjets}
\end{figure}




%-------------------------------------------------------------------------------

%-------------------------------------------------------------------------------
\section{Unfolding}
\label{sec:unfold}
%-------------------------------------------------------------------------------

After subtracting the background from the detector-level distributions, as described in Sec.~\ref{sec:analysis}, the data are corrected for resolution and acceptance effects.  The fiducial volume of the measurement is described by the particle-level object and event selection in Sec.~\ref{sec:obj}.  First, the data are corrected for events that pass the detector-level selection but not the particle-level selection using the simulations introduced in Sec.~\ref{sec:data}.  Then, the iterative Bayes (IB) unfolding technique~\cite{DAgostini:1994zf} is used to correct for the detector resolution in events that pass both the detector-level and particle-level selections.  The IB method is applied with four iterations implemented in the RooUnfold framework~\cite{Adye:2011gm}.  After the application of the response matrix, a final correction is applied to account for events that pass the particle-level but not detector-level selection.  Uncertainties on the unfolding procedure are described in Sec.~\ref{sec:systs}.

%-------------------------------------------------------------------------------
\section{Uncertainties}
\label{sec:systs}
\input{sections/systematics}
%-------------------------------------------------------------------------------

%\clearpage

%-------------------------------------------------------------------------------
\section{Results}
\label{sec:result}
\label{sec:gbb-results}

Figure~\ref{fig:gbb-results} presents a summary of the results. Generally we observe that the Sherpa predictions are closer to data, especially for low \zpt and \mpt. However both generators seem to show an opposite trend in $\dphi$ compared to data. The data distribution is upward curving in contrast to downward curving or flat distributions in MC. Private communication with authors of Pythia and Sherpa reveal that the Vincia generator showed same trend as data distribution. However, given Sherpa and Vincia both included higher order calculations beyond matrix element, it is unclear which feature exactly in Vinciat leads to a better agreement. This is certainly worth investigating in the future. For \drbb, both generators show very good agreement with data. This study suggests further investigations of this kind or others to fully explore the quantities where disagreements between data and MC are prominent and are largely unconstrained. The truth level distributions are available at HEPdata\footnote{http://hepdata.cedar.ac.uk/} for MC tunning or different interpretations. 

\begin{figure}[htpb!]
\begin{center}
  \includegraphics[width=0.45\linewidth]{figures/gbb/unfold_final_dR}
  \includegraphics[width=0.45\linewidth]{figures/gbb/unfold_final_zpt}
  \includegraphics[width=0.45\linewidth]{figures/gbb/unfold_final_mass}
  \includegraphics[width=0.45\linewidth]{figures/gbb/unfold_final_dtheta}
\caption[]{The unfolded data compared with Pythia. The data points have error bars that are the statistical uncertainties and the bands are the total systematic uncertainties. The bands for the Pythia prediction represented by a square indicate the variation of a $\pm10\%$ in the final state shower $\alpha_s$. The additional set of Pythia markers use $m_{b\bar b}^2/4$ for the renormalization scale or turns off gluon polarization.}
\label{fig:gbb-results}
\end{center}
\end{figure}



%-------------------------------------------------------------------------------

\FloatBarrier

%-------------------------------------------------------------------------------
\section{Conclusion}
\label{sec:conclusion}
%-------------------------------------------------------------------------------

This paper presents a measurement of various properties of $g\rightarrow b\bar{b}$ at high $p_\text{T}$ and low $\Delta R(b,b)$ from 33 fb$^{-1}$ of $\sqrt{s}=13$ TeV $pp$ collisions recorded by the ATLAS detector.  A flavor-fraction fit is used to remove contributions from processes other than $g\rightarrow b\bar{b}$.  The fitted fractions significantly disagree with the pre-fit \PYTHIA predictions and suggest that further studies could improve the modeling of analyses sensitive to these fractions.  The measured properties are unfolded to correct for the detector acceptance and resolution for direct comparison to particle-level models.  Comparisons are made at particle level between the distributions and various models of jet formation. Simulations from the \SHERPA event generator generally provide a better model than \PYTHIA, especially for the $\Delta\mathrm{\theta}_\text{ppg,gbb}$ observable which is sensitive to the modeling of the gluon polarization. The particle-level spectra are publicly available~\cite{hepdata} for further interpretation and can be used to validate QCD MC predictions and tune their models' free parameters.

%-------------------------------------------------------------------------------
\section*{Acknowledgements}
%-------------------------------------------------------------------------------

% Acknowledgements for papers with collision data
% Version 14-Feb-2018

% Standard acknowledgements start here
%----------------------------------------------
We thank CERN for the very successful operation of the LHC, as well as the
support staff from our institutions without whom ATLAS could not be
operated efficiently.

We acknowledge the support of ANPCyT, Argentina; YerPhI, Armenia; ARC, Australia; BMWFW and FWF, Austria; ANAS, Azerbaijan; SSTC, Belarus; CNPq and FAPESP, Brazil; NSERC, NRC and CFI, Canada; CERN; CONICYT, Chile; CAS, MOST and NSFC, China; COLCIENCIAS, Colombia; MSMT CR, MPO CR and VSC CR, Czech Republic; DNRF and DNSRC, Denmark; IN2P3-CNRS, CEA-DRF/IRFU, France; SRNSFG, Georgia; BMBF, HGF, and MPG, Germany; GSRT, Greece; RGC, Hong Kong SAR, China; ISF, I-CORE and Benoziyo Center, Israel; INFN, Italy; MEXT and JSPS, Japan; CNRST, Morocco; NWO, Netherlands; RCN, Norway; MNiSW and NCN, Poland; FCT, Portugal; MNE/IFA, Romania; MES of Russia and NRC KI, Russian Federation; JINR; MESTD, Serbia; MSSR, Slovakia; ARRS and MIZ\v{S}, Slovenia; DST/NRF, South Africa; MINECO, Spain; SRC and Wallenberg Foundation, Sweden; SERI, SNSF and Cantons of Bern and Geneva, Switzerland; MOST, Taiwan; TAEK, Turkey; STFC, United Kingdom; DOE and NSF, United States of America. In addition, individual groups and members have received support from BCKDF, the Canada Council, CANARIE, CRC, Compute Canada, FQRNT, and the Ontario Innovation Trust, Canada; EPLANET, ERC, ERDF, FP7, Horizon 2020 and Marie Sk{\l}odowska-Curie Actions, European Union; Investissements d'Avenir Labex and Idex, ANR, R{\'e}gion Auvergne and Fondation Partager le Savoir, France; DFG and AvH Foundation, Germany; Herakleitos, Thales and Aristeia programmes co-financed by EU-ESF and the Greek NSRF; BSF, GIF and Minerva, Israel; BRF, Norway; CERCA Programme Generalitat de Catalunya, Generalitat Valenciana, Spain; the Royal Society and Leverhulme Trust, United Kingdom.

The crucial computing support from all WLCG partners is acknowledged gratefully, in particular from CERN, the ATLAS Tier-1 facilities at TRIUMF (Canada), NDGF (Denmark, Norway, Sweden), CC-IN2P3 (France), KIT/GridKA (Germany), INFN-CNAF (Italy), NL-T1 (Netherlands), PIC (Spain), ASGC (Taiwan), RAL (UK) and BNL (USA), the Tier-2 facilities worldwide and large non-WLCG resource providers. Major contributors of computing resources are listed in Ref.~\cite{ATL-GEN-PUB-2016-002}.
%----------------------------------------------



%-------------------------------------------------------------------------------
\clearpage
%\appendix
%\part*{Appendix}
%\addcontentsline{toc}{part}{Appendix}
%-------------------------------------------------------------------------------

%In a paper, an appendix is used for technical details that would otherwise disturb the flow of the paper.
%Such an appendix should be printed before the Bibliography.

%-------------------------------------------------------------------------------
% If you use biblatex and either biber or bibtex to process the bibliography
% just say \printbibliography here
\printbibliography
% If you want to use the traditional BibTeX you need to use the syntax below.
%\bibliographystyle{bib/bst/atlasBibStyleWoTitle}
%\bibliography{atlas-document,bib/ATLAS,bib/CMS,bib/ConfNotes,bib/PubNotes}
%-------------------------------------------------------------------------------

%-------------------------------------------------------------------------------
% Auxiliary material - comment out the following line if you do not have any
\include{atlas-document-auxmat}
%-------------------------------------------------------------------------------

\clearpage

\begin{figure}[htbp]
  \centering
 \includegraphics[width=0.48\textwidth]{figures/Canv_Fit_dphi_LpT_INF_SpT_INF_coarse_x}
 \includegraphics[width=0.48\textwidth]{figures/Canv_Fit_b0_25_dphi_0_3_LpT_INF_SpT_INF_coarse_y}\\
\caption{The distribution of $\subsdzero$ in simulation, post-fit, for the higher $p_\text{T}$ track-jet (left) and for the lower $p_\text{T}$ track-jet (right) in the bin $0<\Delta\mathrm{\theta}_\text{ppg,gbb}<\pi/5$. The three components are the signal double-$b$ (`BB'), the background single $b$ (`B'), and the background non-$b$ components (`L+C').  Percentages reported in the legend indicate the pre- and post-fit fraction of each component.  Only data and MC statistical uncertainties are shown.}
  \label{fig:fits2}
\end{figure}

\begin{figure}[htbp]
  \centering
 \includegraphics[width=0.48\textwidth]{figures/Canv_Fit_zpt_LpT_INF_SpT_INF_coarse_x}
 \includegraphics[width=0.48\textwidth]{figures/Canv_Fit_b0_25_zpt_0_3_LpT_INF_SpT_INF_coarse_y.pdf}\\
\caption{The distribution of $\subsdzero$ in simulation, post-fit, for the higher $p_\text{T}$ track-jet (left) and for the lower $p_\text{T}$ track-jet (right) in the bin $0.0<z(p_\text{T})<0.1$. The three components are the signal double-$b$ (`BB'), the background single $b$ (`B'), and the background non-$b$ components (`L+C').  Percentages reported in the legend indicate the pre- and post-fit fraction of each component.  Only data and MC statistical uncertainties are shown.}
  \label{fig:fits3}
\end{figure}

\begin{figure}[htbp]
  \centering
 \includegraphics[width=0.48\textwidth]{figures/Canv_Fit_M_LpT_INF_SpT_INF_coarse_x}
 \includegraphics[width=0.48\textwidth]{figures/Canv_Fit_b0_25_M_0_3_LpT_INF_SpT_INF_coarse_y}\\
\caption{The distribution of $\subsdzero$ in simulation, post-fit, for the higher $p_\text{T}$ track-jet (left) and for the lower $p_\text{T}$ track-jet (right) in the bin $-3.0<\log(m/p_\text{T})<-2.2$. The three components are the signal double-$b$ (`BB'), the background single $b$ (`B'), and the background non-$b$ components (`L+C').  Percentages reported in the legend indicate the pre- and post-fit fraction of each component.  Only data and MC statistical uncertainties are shown.}
  \label{fig:fits4}
\end{figure}

%-------------------------------------------------------------------------------
% Extra tables etc. for HepData - comment in the following line if you have any
% \include{atlas-document-hepdata}
%-------------------------------------------------------------------------------

\end{document}
